\usepackage[utf8]{inputenc}
\usepackage[T1]{fontenc}
\usepackage{amsmath,amsfonts,amssymb,amsthm}
\usepackage[british,polish]{babel}
\usepackage{courier}
\usepackage{etoolbox}
\usepackage{lmodern}
\usepackage{graphicx} 
\usepackage{hyperref}
\usepackage{booktabs}
\usepackage{tikz}

%%%% pgfplots
\usepackage{pgfplots}
\pgfplotsset{compat=1.9}

%%%%
\usepackage{mathtools}
\usepackage{geometry}

%%%% dodatki
\usepackage[page]{appendix}
\renewcommand{\appendixtocname}{Dodatki}
\renewcommand{\appendixpagename}{Dodatki}
\renewcommand{\appendixname}{Dodatek}

%%%% spacingi
\usepackage{setspace}
\onehalfspacing
\frenchspacing

\usepackage{listings}
%%%% listingi
\usepackage{listings}
\lstset{%
language=C++,%
commentstyle=\textit,%
identifierstyle=\textsf,%
keywordstyle=\sffamily\bfseries, %\texttt, %
%captionpos=b,%
tabsize=3,%
frame=lines,%
numbers=left,%
numberstyle=\tiny,%
numbersep=5pt,%
breaklines=true,%
% morekeywords={aaa,bbb},%
escapeinside={@*}{*@},%
}

%%%% zaawansowane formatowanie
\usepackage{fancyhdr}
\pagestyle{fancy}
\fancyhf{}
\fancyhead[LO]{\nouppercase{\it\rightmark}}
\fancyhead[RE]{\nouppercase{\it\leftmark}}
\fancyhead[LE,RO]{\it\thepage}

\fancypagestyle{tylkoNumeryStron}{%
   \fancyhf{} 
   \fancyhead[LE,RO]{\it\thepage}
}

\fancypagestyle{NumeryStronNazwyRozdzialow}{%
   \fancyhf{} 
   \fancyhead[LO]{\nouppercase{\it\rightmark}}
   \fancyhead[RE]{\nouppercase{\it\leftmark}}
   \fancyhead[LE,RO]{\it\thepage}
}


%%%% stylowanie języków obcych 
\newcommand{\foreign}[1]{\emph{#1}}
\newcommand{\eng}[1]{{\selectlanguage{british}\foreign{#1}}}

%%%% coś?
\newcounter{stronyPozaNumeracja}
\newcommand{\hcancel}[1]{%
    \tikz[baseline=(tocancel.base)]{
        \node[inner sep=0pt,outer sep=0pt] (tocancel) {#1};
        \draw[red] (tocancel.south west) -- (tocancel.north east);
    }
}

%%%% polskie miesiące
\newcommand{\miesiac}{
  \ifcase\the\month
  \or styczeń
  \or luty
  \or marzec
  \or kwiecień
  \or maj
  \or czerwiec
  \or lipiec
  \or sierpień
  \or wrzesień
  \or październik
  \or listopad
  \or grudzień
  \fi}

%%%% Helvetica font macros na str tytuowej:
\newcommand{\headerfont}{\fontfamily{phv}\fontsize{18}{18}\bfseries\scshape\selectfont}
\newcommand{\titlefont}{\fontfamily{phv}\fontsize{18}{18}\selectfont}
\newcommand{\otherfont}{\fontfamily{phv}\fontsize{14}{14}\selectfont}

%%%%%%%%%%%%%%%%%%%%%%%%%%%%%%%%%%%%%%%%%%%%%%%%%%

\newcommand{\autor}{Grzegorz Kazana}
\newcommand{\promotor}{dr inż. Mariusz Boryczka}
\newcommand{\tytul}{Zastosowanie optymalizacji mrowiskowej do ukrywania danych w obrazach cyfrowych}
\newcommand{\unisl}{Uniwersytet Śląski}
\newcommand{\wydzial}{Wydział Nauk Ścisłych i Technicznych}