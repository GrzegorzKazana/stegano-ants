
\chapter*{Wstęp}\label{chap:intro}\addcontentsline{toc}{chapter}{Wstęp}
{
    % - przesyłamy coraz więcej informacji przez internet
    % - wiele z tych informacji jest poufnych
    % - dlaczego potrzebujemy ukrywać dane
    % - dlaczego potrzebujmy lepszych rozwiązań, co można osiągnąć
    Bezpieczeństwo oraz poufność komunikacji od zawsze była niezwykle istotną dziedziną ludzkiego życia. Historycznie,
    potrzeba niejawnej wymiany informacji dotyczyła głównie władców i polityków. W ostatnich dekadach zaczęło się to
    zmieniać. W związku z wszechobecną cyfryzacją życia codziennego i rozprzestrzenieniem się komunikacji internetowej,
    problem poufnej wymiany danych dotyczy każdego. Oprócz metod szyfrowania treści przekazu, rozsądnym krokiem wydają
    się również próby utajenia faktu istnienia komunikacji. Tym problemem zajmuje się steganografia. Drastycznie rosnąca
    objętość danych przesyłanych przez sieci komunikacyjne otwiera nowe możliwości i daje nadzieje na opracowanie metod,
    które poprzez wykorzystanie dużej objętości jawnych danych nośnych pozwolą skutecznie ukryć poufny przekaz.

    Celem pracy było zbadanie możliwości wykorzystania optymalizacji mrowiskowej w zastosowaniach steganograficznych
    oraz ocenienie jej efektywności. Podczas analizy zbadane zostało zastosowanie systemów mrówkowych w procesie doboru
    obszarów obrazów cyfrowych, pozwalających na ukrycie największej ilości informacji przy jednoczesnej najmniejszej
    stracie jakości obrazu.

    Do zakresu pracy należy przegląd istniejącej literatury, zaproponowanie i porównanie metod grafowej reprezentacji
    obrazu oraz interpretacji śladu feromonowego. Następnie została przeprowadzona analiza uzyskanych rezultatów pod
    kątem utraty jakości przy założonej ilości ukrytych danych. Ostatnim krokiem było odniesienie rezultatów do
    powiązanych tematycznie prac.

    W pierwszym rozdziale przytoczono krótką historię steganografii, opisano jej różne zastosowania oraz metody.
    Największą uwagę poświęcono steganografii cyfrowej, wykorzystującej jako medium obrazy. Przytoczono główne
    założenia, wady i zalety metod przestrzennych oraz częstotliwościowych.

    Drugi rozdział poświęcono obszernemu zagadnieniu algorytmów mrówkowych. Omówiono ich genezę, zasadę działania oraz
    najważniejsze zastosowania. Zawarto również szczegółowy opis najpopularniejszych wariantów systemów mrówkowych.

    Rozdział trzeci stanowi rozwinięcie dwóch poprzedzających i omawia możliwość zastosowania systemów mrówkowych w
    stegnografii. Zaproponowano i opisano autorską metodę wykorzystującą system mrówkowy do wyznaczania złożonych
    obszarów obrazu, w których hipotetycznie można umieścić największą ilość danych bez postrzegalnej utraty jakości.

    Tematem rozdziału czwartego jest omówienie zaimplementowanego narzędzia umożliwiającego ukrywanie danych w obrazach
    przy wykorzystaniu systemu mrówkowego za pomocą metod opisanych w poprzedzającym rozdziale.

    W rozdziale piątym opisano metodę oceny uzyskanych rezultatów oraz odniesiono się do wyników przytoczonych w
    powiązanych tematycznie pracach.
    %
}
