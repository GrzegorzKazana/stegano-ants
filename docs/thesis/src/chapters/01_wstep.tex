
\chapter*{Wstęp}\label{chap:intro}\addcontentsline{toc}{chapter}{Wstęp}
{
    % - przesyłamy coraz więcej informacji przez internet
    % - wiele z tych informacji jest poufnych
    % - dlaczego potrzebujemy ukrywać dane
    % - dlaczego potrzebujmy lepszych rozwiązań, co można osiągnąć

    % \section{Cel i zakres pracy}
    % {
        % - zaproponowanie metody pozwalającej na ukrywanie danych przy jednoczesnej
        %   niskiej wykrywalności i subiektywnej degradacji jakości obrazu
    % }

    % \section{Struktura pracy}
    % {
        % - co zrobiono w którym rozdziale
    % }

    Celem pracy jest zbadanie możliwości wykorzystania optymalizacji mrowiskowej w zastosowaniach steganograficznych
    oraz ocenienie jej efektywności. Podczas analizy badane będzie zastosowanie systemów mrówkowych w procesie doboru
    obszarów obrazów cyfrowych, pozwalających na ukrycie największej ilości informacji przy jednoczesnej najmniejszej
    stracie jakości obrazu.

    Do zakresu pracy należy przegląd istniejącej literatury, zaproponowanie i porównanie metod grafowej reprezentacji
    obrazu oraz interpretacji śladu feromonowego. Następnie zostanie przeprowadzona analiza uzyskanych
    rezultatów pod kątem ilości ukrytych informacji przy założonej utracie jakości. Ostatnim krokiem jest odniesienie
    rezultatów do powiązanych tematycznie prac.

    W pierwszym rozdziale przytoczono krótką historię steganografii, opisano jej różne zastosowania oraz metody.
    Największą uwagę poświęcono steganografii cyfrowej, wykorzystującej jako medium obrazy. Przytoczono główne
    założenia, wady i zalety metod przestrzennych oraz częstotliwościowych.

    Drugi rozdział poświęcono obszernemu zagadnieniu algorytmów mrówkowych. Omówiono ich genezę, zasadę działania oraz
    najważniejsze zastosowania. Zawarto również szczegółowy opis najpopularniejszych wariantów systemów mrówkowych.

    Rozdział trzeci stanowi rozwinięcie dwóch poprzedzających i omawia możliwość zastosowania systemów mrówkowych w
    stegnografii. Zaproponowano i opisano autorską metodę wykorzystującą system mrówkowy do wyznaczania złożonych
    obszarów obrazu, w których hipotetycznie można umieścić największą ilość danych bez postrzegalnej utraty jakości.

    Tematem rozdziału czwartego jest omówienie zaimplementowanego narzędzia umożliwiającego ukrywanie danych w obrazach
    przy wykorzystaniu systemu mrówkowego za pomocą metod opisanych w poprzedzającym rozdziale.

    W rozdziale piątym opisano metodę oceny uzyskanych rezultatów oraz odniesiono się do wyników przytoczonych w
    powiązanych tematycznie pracach.
    %
}


%%% Odwołanie do innych rozdziałów
% Treścią \ref{chap:analysis}. rozdziału jest ...
