\chapter{Zastosowanie systemów mrówkowych w steganografii}\label{chap:antsys}
{
    % - wstęp
    %   - skoro mrówki mają tak szerokie zastosowania, to czemu nie użyć ich w steganografii?
    %   - jaki jest główny cel oraz problem steganografii? (jakoś obrazu jest odwrotnie proporcjonalna do ilości
    %     ukrytych informacji)
    %   - pomysł: odpowiedni dobór miejsc w których zostaną ukryte informacje (wykorzystanie `complex` regions)
    %   - zalety: stochastyczność procesu (bezpieczeństwo++)
    % - przytoczenie (krótkie) istniejących prac, ich założeń/wad/zalet
    %%  moja metoda:
    % - kwestia reprezentacji obrazu jako grafu
    %   - czym są węzły
    %   - czym są krawędzie
    %   - jak wyznaczono długości krawędzi
    % - modyfikacje zasad działania systemu mrówkowego aby przystosować go do problemu
    %   - brak pełnych cykli (ograniczenie ilości kroków)
    % - interpretacja śladu feromonowego
    %   - metody przekształceń (tzn ograniczenie zakresu wartości, skalowanie itp)
}

% ALG MRÓWKOWE W STEGANOGRAFII
% file:///Users/grzegorzkazana/Desktop/Stegano_Ant/Ant%20colony%20optimization%20with%20horizontal%20and%20vertical%20crossover%20search%20Fundamental%20visions%20for%20multi-thres.image%20segmentation.pdf
% file:///Users/grzegorzkazana/Desktop/Stegano_Ant/A%20Novel%20Technique%20for%20Steganography%20Method%20Based%20on.pdf
% file:///Users/grzegorzkazana/Desktop/Stegano_Ant/Ant%20Colony%20Optimization%20ACO%20based%20Data%20Hiding%20in%20Image%20Complex%20Region.pdf
% file:///Users/grzegorzkazana/Desktop/Stegano_Ant/Ant%20Colony%20Optimization%20To%20Enhance%20Image%20Steganography.pdf
% file:///Users/grzegorzkazana/Desktop/Stegano_Ant/HIGH%20CAPACITY%20AND%20OPTIMIZED%20IMAGE%20STEGANOGRAPHY%20TECHNIQUE%20BASED%20ON%20ANT%20COLONY%20OPTIMIZATION%20ALGORITHM.pdf
% file:///Users/grzegorzkazana/Desktop/Stegano_Ant/New%20steganography%20algorithm%20to%20conceal%20a%20large%20amount%20of%20secret%20message%20using%20hybrid%20adaptive%20neural.pdf
