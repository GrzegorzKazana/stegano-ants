\chapter{Zastosowanie systemów mrówkowych w steganografii}\label{chap:stegoants}
{
    % - wstęp
    %   - skoro mrówki mają tak szerokie zastosowania, to czemu nie użyć ich w steganografii?
    %   - jaki jest główny cel oraz problem steganografii? (jakoś obrazu jest odwrotnie proporcjonalna do ilości
    %     ukrytych informacji)
    %   - pomysł: odpowiedni dobór miejsc w których zostaną ukryte informacje (wykorzystanie `complex` regions)
    %   - zalety: stochastyczność procesu (bezpieczeństwo++)
    % - przytoczenie (krótkie) istniejących prac, ich założeń/wad/zalet
    %%  moja metoda:
    % - coś powiedzieć o zasadzie Kerckhoffsa i dodatkowym bezpieczeństwie z nią związaną
    % - kwestia reprezentacji obrazu jako grafu
    %   - czym są węzły
    %   - czym są krawędzie
    %   - jak wyznaczono długości krawędzi
    % - modyfikacje zasad działania systemu mrówkowego aby przystosować go do problemu
    %   - brak pełnych cykli (ograniczenie ilości kroków)
    % - interpretacja śladu feromonowego
    %   - metody przekształceń (tzn ograniczenie zakresu wartości, skalowanie itp)

    Jak przedstawiono w poprzednich rozdziałach, zagadnienia steganografii oraz systemów mrówkowych są złożone, nawet
    gdy są analizowane w izolacji. Obie dziedziny mogą zostać powiązane, jeśli spojrzymy na nie jako na parę problemu
    optymalizacji oraz metaheurystykę mogącą posłużyć do jego rozwiązania.

    % GŁÓWNY PROBLEM: ILOŚĆ vs JAKOŚĆ
    Wykorzystanie systemów steganograficznych jest bezpośrednio związane z jednoczesną realizacją celów stojących sobie
    w opozycji. Idealny system steganograficzny, niezależnie od stosowanego medium nośnego, cechuje się jednocześnie
    dużą pojemnością, odpornością na zakłócenia i zniekształcenia nośnika i niską wykrywalnością istnienia przekazu.
    Intuicyjnie, lecz również na podstawie wszelkich obserwacji, można zauważyć że jednoczesna maksymalizacja ukrywanego
    przekazu powoduje wzrost cech niepożądanych, takich jak wzrost podatności na ataki steganograficzne. Oznacza to, że
    kluczowym aspektem wszystkich metod steganograficznych jest zachowanie balansu pomiędzy realizacją tych celów. W
    zastosowaniach o krytycznej poufności istotniejsze będzie zapewnienie niższej wykrywalności, nawet jeśli się
    odbędzie kosztem stosunku wiadomości do sygnału. W zależności od wybranej metody, kontrolowanie tego balansu może
    się to sprowadzać do regulacji parametrów algorytmu lub wyboru nośnika o odpowiednio dużym rozmiarze.

    % METODY HEURYSTYCZNE - ZALETY W STEGANOGRAFII
    Nie oznacza to jednak, że należy zaniechać poszukiwań metod pozwalających na zachowanie zadowalającego poziomu
    każdego z parametrów procesu. W rzeczywistości, zagadnienie wykorzystania metaheurystyk oraz metod uczenia
    maszynowego do celów steganografii jest aktywną dziedziną odnoszącą ciągłe sukcesy\cite{ElEmam2013NewSA,
    Zhao2021AntCO, Khan2018AntCO, Saleema2016ANS, Li2007ASM}.
    % STOSOWANIE KLUCZY
    Kolejnym aspektem przemawiającym na korzyść stosowania metod heurystycznych do celów steganograficznych jest ich
    stochastyczny charakter. Podstawowym problemem najprostszych technik steganograficznych, takich jak \textit{LSB},
    jest sekwencyjność wyboru fragmentów nośnika informacji. W przypadku ukrywania informacji w obrazach,
    najoczywistszym działaniem jest kodowanie danych w kolejnych pikselach obrazu. Takie podejście jest wyjątkowo
    wrażliwe na najprostsze formy ataków, za równo steganograficznych jak i statystycznych. W celu uniknięcia powyższej
    podatności, możliwe jest stosowanie dodatkowych kluczy instruujących, które i w jakiej kolejności segmenty nośnika
    należy analizować\cite{AlHusainy2019ASI}. Wykorzystanie stochastycznych procesów które zachodzą w znacznej części
    metaheurystyk, automatycznie rozwiązuje problem przewidywalności umiejscowienia informacji. Gwarantuje również
    możliwość powtarzalnego odtworzenia jego działania poprzez ustalanie ziarna generatora liczb losowych. W takim
    przypadku, jego wartość stanowi swoisty klucz zwiększający bezpieczeństwo ukrytych danych.
    % ZASADA KERCKHOFFSA
    Dodatkowym argumentem przemawiającym za słusznością wykorzystania klucza w metodach steganograficznych, jest słynna
    zasada Kerckhoffsa\cite{Guillot2013AugusteKE}, będąca fundamentem współczesnej kryptografii. Jej treść głosi, że
    bezpieczeństwo systemu powinno zostać zachowane, nawet jeśli osoba próbująca odkryć poufne informacje zna wszystkie
    szczegóły jego działania. Gwarantem bezpieczeństwa musi być prywatny klucz. Zasada ta stoi w opozycji do systemów
    opierających się na niejawności \textit{(ang. Security by obscurity)}.

    % SYSTEMY MRÓWKOWE W STEGANOGRAFII
    W związku z zaletami metod heurystycznych, oraz nadziejami na znalezienie sposobu optymalizacji przeciwstawnych cech
    steganogramów, próby wykorzystania systemów mrówkowych i mrowiskowych do ukrywania danych w obrazach były
    niejednokrotnie podejmowane\cite{Priya2018HIGHCA, ZghaerACOStegEN, Khan2018AntCO}.
    % REPREZENTACJA PROBLEMU + INTERPRETACJA WYNIKÓW
    Najważniejszym aspektem charakteryzującym każdą z opisanych metod jest sposób reprezentacji problemu. W celu
    zastosowania metaheurystyki systemu mrówkowego, konieczne jest przedstawienie danych wejściowych w postaci grafu.
    Wybór dotyczący zasady jego budowy ma fundamentalny wpływ na uzyskiwane rezultaty oraz efektywność algorytmu.
    Kolejnym kluczowym zagadnieniem jest interpretacja rezultatów pracy wirtualnych mrówek. W tej kwestii, istnieją
    conajmniej dwie obierane ścieżki przez eksperymentatorów. Za wynik działania algorytmu można uznać najkrótszą bądź
    najpopularniejszą ścieżkę obieraną przez mrówki - takie podejście oznacza pozyskanie dyskretnej listy wykorzystanych
    krawędzi. Alternatywnie, jako rezultat można uznać utworzony ślad feromonowy. Korzystając z tego podejścia,
    utrzymujemy rozmyty zbiór krawędzi reprezentujących najlepszą ścieżkę - krawędzie częściej należące do krótszych
    rozwiązań problemu komiwojażera będą bardziej do niego należeć niż ścieżki rzadko obierane. Analiza rozmytego zbioru
    rozwiązań pozostawia szerszą możliwość interpretacji wyników. Dodatkowo, charakterystyka uzyskiwanego śladu
    feromonowego jest bardziej podatna na zmiany rodzaju systemu mrówkowego oraz jego parametry.

    % PRZYKŁD - CZĘSTOTLIWOŚCIOWO
    Systemy mrowiskowe mogą zostać wykorzystane za równo w steganografii przy pomocy obrazów cyfrowych w dziedzinie
    przestrzennej\cite{ZghaerACOStegEN, Khan2018AntCO}, jak i częstotliwościowej\cite{Priya2018HIGHCA}. W artykule
    \textit{High capacity and optimized image steganography technique based on ant colony optimization algorithm},
    zaproponowano metodę opartą o przytoczoną metodę całkowitoliczbowej transformaty falkowej (\textit{IWT}). Twórcy
    opisali algorytm wykorzystujący system mrowiskowy do ukrycia sekretnej informacji w współczynnikach dziedziny
    transformaty. Przeprowadzone eksperymenty wykazały wysoką skuteczność metody\cite{Priya2018HIGHCA}.
    % PRZYKŁD - PRZESTRZENNIE
    Pomimo że pozostałe przestudiowane prace oparte są o analizę obrazu w technice przestrzennej, sposób reprezentacji
    problemu i interpretacji wyników jest zdecydowanie odmienny. W pracy \textit{Ant Colony Optimization To Enhance
    Image Steganography}\cite{ZghaerACOStegEN}, obrazy zostały podzielone na bloki o rozmiarach $2\times2$ lub
    $5\times5$. Każdy blok jest interpretowany jako graf, w którym wierzchołkami są piksele, a długościami krawędzi są
    odwrotności błędu średniokwadratowego spowodowanego przez ukrycie w danym pikselu jednego bitu informacji. Uzyskana
    przez system mrówkowy ścieżka wskazuje, w których pikselach należy umieścić informację aby uzyskać najmniejszy wpływ
    na różnicę między obrazem nośnym a stganogramem\cite{ZghaerACOStegEN}.
    % PRZYKŁD - PRZESTRZENNIE 2
    Przykładem pracy wykorzystującej wartości śladu feromonowego naniesionego przez mrówki jest \textit{Ant Colony
    Optimization (ACO) based Data Hiding in Image Complex Region}\cite{Khan2018AntCO}. W powyższej pracy, mrówki
    poruszają się po grafie zbudowany, na podstawie bitmapy nośnej. Jego wierzchołkami są piksele, a długościami
    krawędzi różnice intensywności pikseli przez nie łączone. Dążeniem stosowania takiej reprezentacji jest wykrycie
    krawędzi oraz złożonych obszarów obrazu, pozwalających na ukrycie większej ilości informacji przy jednoczesnej
    niższej wykrywalności ingerencji w obraz. Po ukończeniu pracy systemu mrowiskowego, ustalana jest wartość graniczna
    feromonu, dla której piksele uznawane są za należące do złożonego segmentu. W ten sposób, obraz zostaje podzielony
    na dwa zbiory pikseli, w tych związanych z wartością feromonu przekraczającą wartość graniczną zostaje stosowana
    technika \textit{LSB}. Pozostałe piksele pozostają niezmienione. Za pomocą wyników eksperymentów udaje się dowieźć,
    jest to skuteczna metoda. Jej dodatkową zaletą jest możliwość parametryzacji i zmiany sposobu wyznaczania granicznej
    wartości feromonu, co pozwala na kontrolę pojemności steganogramu kosztem jakości\cite{Khan2018AntCO}.

    % Priya2018HIGHCA - ACO + IWT, ants used to select coefficients in which to embed data
    % ZghaerACOStegEN - aco/mmas + przestrzenne, podział na bloki 2x2 lub 5x5 + ukrywanie bitów w blokach tak aby
    % minimalizować MSE. 1 pixel = 1 node
    % Khan2018AntCO - aco + przestrzenne, 1 pixel = 1 wierzchołek, ślad feromonowy pozwala na decyzję czy piksel należy
    % do `complex` region czy nie (binarnie). W `complex` region ukrywamy dane LSB. Podział obrazu na 2 grupy

}

% ALG MRÓWKOWE W STEGANOGRAFII
% Zhao2021AntCO file:///Users/grzegorzkazana/Desktop/Stegano_Ant/Ant%20colony%20optimization%20with%20horizontal%20and%20vertical%20crossover%20search%20Fundamental%20visions%20for%20multi-thres.image%20segmentation.pdf
% Mohadeseh2013ANT file:///Users/grzegorzkazana/Desktop/Stegano_Ant/A%20Novel%20Technique%20for%20Steganography%20Method%20Based%20on.pdf
% Khan2018AntCO file:///Users/grzegorzkazana/Desktop/Stegano_Ant/Ant%20Colony%20Optimization%20ACO%20based%20Data%20Hiding%20in%20Image%20Complex%20Region.pdf
% ZghaerACOStegEN file:///Users/grzegorzkazana/Desktop/Stegano_Ant/Ant%20Colony%20Optimization%20To%20Enhance%20Image%20Steganography.pdf
% Priya2018HIGHCA file:///Users/grzegorzkazana/Desktop/Stegano_Ant/HIGH%20CAPACITY%20AND%20OPTIMIZED%20IMAGE%20STEGANOGRAPHY%20TECHNIQUE%20BASED%20ON%20ANT%20COLONY%20OPTIMIZATION%20ALGORITHM.pdf
% ElEmam2013NewSA file:///Users/grzegorzkazana/Desktop/Stegano_Ant/New%20steganography%20algorithm%20to%20conceal%20a%20large%20amount%20of%20secret%20message%20using%20hybrid%20adaptive%20neural.pdf
