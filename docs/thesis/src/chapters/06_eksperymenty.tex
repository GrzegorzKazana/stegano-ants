\chapter{Wyniki eksperymentów}\label{chap:results}
{
    % co i po co
    %
    % jak przeprowadzano eksperymenty
    %   - na jakich obrazach (RGB, rozdzielczość)
    %   - jakie ukrywane dane (tekst)
    %   - narzucanie pojemności obrazu w celu sprawdzenia jakości
    %   - stałe parametry algorytmu (jeśli są (?))
    %
    % miary jakości i ich definicje
    %   - do czego służą
    %   - wzór
    %   - jak oddają ludzką percepcję
    %      - błąd średniokwadratowy (MSE)
    %      - szczytowy stosunek sygnału do szumu (PSNR)
    %      - podobieństwo strukturalne (SSIM)
    %
    % wyniki eksperymentów
    %   - metoda oparta o wierzchołki
    %       - omówienie parametrów algorytmu (czemu/jakie/w jakim zakresie badano)
    %       - wyniki dla różnych parametrów
    %       - krótkie wnioski
    %           - czy wpływ parametrów pokrywa się z intuicją
    %           - które systemy (dispatcher/updater) się lepiej sprawdziły, czemu
    %   - metoda oparta o krawędzie
    %       - omówienie parametrów algorytmu (czemu/jakie/w jakim zakresie badano)
    %       - omówienie metod segmentacji
    %       - wyniki dla różnych parametrów
    %       - krótkie wnioski
    %           - wpływ parametrów
    %           - wpływ ilości segmentów (krawędzi)
    %           - które systemy (dispatcher/updater) się lepiej sprawdziły, czemu
    %
    % ocena subiektywna (widzę/nie widzę)
    %   - czy pokrywa się z miarami liczbowymi
    %   - od jakiej ilości danych są widoczne
    %
    % porównanie z innymi pracami (?)
    %   - znaleźć popularne obrazy i porównać dla nich wyniki
    %   - (?) porównanie z zwykłym LSB
    %   - (?) porównanie z vLSB opartym o proste wykrywanie krawędzi

    % WSTĘP - CO TESTOWANO
    Aby zbadać skuteczność i efektywność metod zaproponowanych w rozdziale \ref{chap:stegoants} oraz podrozdziale
    \ref{sec:method}, postanowiono przeprowadzić pewne eksperymenty. Ich celem była walidacja fundamentalnych założeń,
    takich jak słuszność doboru złożonych sekcji obrazów, sprawdzenie zasadności sposobów wyznaczania reprezentacji
    grafowej problemu i interpretacji śladu feromonowego, numeryczna ocena degradacji jakości steganogramu oraz
    subiektywna opinia dotycząca postrzegalności tych zmian. Numeryczna ewaluacja degradacji jakości obrazu pozwoli
    odnieść uzyskane wyniki do istniejących rozwiązań oraz stwierdzić, czy zaproponowane metody mają swoje zastosowanie.

    % METODA BADAWCZA
    Weryfikacja polegała na przeprowadzeniu procesu ukrywania danych w bitmapach powszechnie wykorzystywanych w
    dziedzinie przetwarzania obrazów.

    % OBRAZY
    Pomimo istnienia alternatywnych zbiorów składających się z syntetycznie
    spreparowanych obrazów mających na celu wyszczególnienie pewnych cech\cite{Uhlmann2018ACI}, zdecydowano się na
    wykorzystanie prawdziwych zdjęć, gdyż bardziej odpowiadają one praktycznym zastosowaniom steganografii. W ich
    doborze kierowano się głównie ilością odniesień w tematycznie powiązanych pracach oraz bieżącymi tendencjami w
    kwestii wykorzystywania publicznych zbiorów obrazów\cite{NoteOnLena1, NoteOnLena2}. Ostatecznie, zdecydowano się
    skupić na obrazach \textit{Mandrill (a.k.a Baboon)}, \textit{Airplane (F-16) (a.k.a Jet)} oraz \textit{Peppers},
    które są udostępniane przez \textit{Uniwersytet Południowej Kalifornii (USC)}\cite{USCDatabase}.
    %% TODO: tutaj obrazki baboon, f16, peppers

    % DANE
    Dane które ukrywano w obrazach miały postać tekstu w formacie \textit{ASCII}, lecz po drobnych adaptacjach metody
    ukrywania danych w obrazie możliwe byłoby ukrywanie dowolnych danych w postaci binarnej. W eksperymentach
    wykorzystano automatycznie wygenerowany tekst \textit{Lorem ipsum} o rozmiarze $625kB$. Cechą zaproponowanej metody
    oraz zaimplementowanego programu jest możliwość skalowania wygenerowanej macierzy maskującej w taki sposób, aby
    można było umieścić zadaną ilość bitów tekstu. Pozwoli to na zbadanie degradacji jakości w zależności od objętości
    ukrywanego tekstu, oraz ułatwi porównanie rezultatów z innymi metodami.

    % BADANE PARAMETRY
    Podczas przeprowadzanych eksperymentów, badano wpływ poszczególnych metod każdego z etapów procesu oraz ich
    parametrów. Porównywane wartości parametrów dotyczą poniższych procesów.

    \begin{enumerate}
        \item Konstrukcji grafu oraz wizualizacji śladu feromonowego.
        \begin{itemize}
            \item Metoda oparta o wierzchołki. Jedynym parametrem jest opcjonalny parametr skalujący $s_0$ obraz
            wejściowy podczas budowania grafu oraz budowania macierzy maskującej. Jego wartości znajdują się w zakresie
            $[0, 1]$. Jego domyślna wartość wynosi $1$, i oznacza budowanie grafu o liczbie wierzchołków równej $w
            \times h$.

            \item Metoda oparta o krawędzie. Do jej parametrów należy algorytm segmentacji oraz docelowa liczba
            segmentów $N_s$ związana z ilością krawędzi grafu. Podczas badań wykorzystano algorytmy prostego podziału na
            na prostokąty, algorytm k-średnich oraz algorytm \textit{SLIC} służący do konstrukcji superpikseli.
        \end{itemize}

        \item Wyznaczenie śladu feromonowego przez różne rodzaje systemów mrówkowych. Do parametrów wspólnych dla
        każdego z wykorzystanych systemów należy:
        \begin{itemize}
            \item ilość mrówek $a$, która domyślnie jest równa ilości wierzchołków grafu $V$,
            \item liczba wykonanych cykli $C$
            \item liczba kroków wykonywanych przez mrówki w każdej iteracji algorytmu $S$, dla grafów skonstruowanych na
            podstawie segmentacji obrazu wynosi jest ona równa ilości wierzchołków $V$
            \item preferencja względem śladu feromonowego $\alpha$
            \item preferencja względem widoczności wierzchołka $\beta$
            \item początkowe natężenie śladu feromonowego $\tau_0$
            \item współczynnik opisujący prędkość wyparowania śladu feromonowego $\rho$
        \end{itemize}

        Do porównanych rodzajów systemów należą poniższe odmiany. Z niektórymi z nich związane są dodatkowe parametry
        oraz domyślne wartości powyższych atrybutów.
        \begin{itemize}
            \item model feromon stały,
            \item model feromon średni,
            \item model feromon cykliczny,
            \item system mrowiskowy, który wprowadza prawdopodobieństwo eksploatacji najkrótszej krawędzi $q_0$ oraz
            ustalona parametr $\alpha = 1$,
            \item system max-min, który wprowadza ograniczenia wartości śladu feromonowego $[\tau_{min}, \tau_{max}]$.
            Ich wartości wyznaczane są za pomocą estymaty dotyczącej długości poszukiwanego cyklu. Wartość początkowa
            feromonu wynosi $\tau_0 = \tau_{max}$.
        \end{itemize}
    \end{enumerate}

    % WYKORZYSTANE MIARY JAKOŚCI
    W celu umożliwienia porównywania jakości steganogramów, zdecydowano się skorzystać z metryk służących
    do porównawczej analizy obrazów. Zastosowane metryki można podzielić na dwie kategorie, obiektywne i subiektywne.
    Metryki obiektywne służą do wyznaczenia pewnej wartości charakteryzującej różnicę pomiędzy dwoma sygnałami. W
    przypadku metryk subiektywnych, ich zadaniem jest również wyznaczenie wartości numerycznej, lecz nacisk kładziony
    jest na korelację wartości z postrzegalnością zmian przez ludzki układ wzrokowy. Do zastosowanych metryk należą:

    % MSE
    Błąd średniokwadratowy (ang. \textit{Mean Square Error, MSE}). Jest jedną z najprostszych metryk służących do
    pomiaru różnic między obrazami. Jej wartości należą do zbioru nieujemnych liczb rzeczywistych, a wartości bliższe
    zeru stanowią o mniejszym spadku jakości. Do jej zalet należy prostota implementacji i możliwość optymalnej
    implementacji. Jedną z jej wad jest niska korelacja z postrzeganiem różnic między obrazami przez ludzi oraz
    nieuwzględnianie informacji o relacji natężenia szumu do wartości sygnału. Jej wartość wyraża wzór \ref{eqt:mse}, w
    którym $\overline{X}$ oznacza wartość średnią.

    \begin{equation}\label{eqt:mse}
        MSE = \frac{1}{n} \sum_{i=1}^n (X_i - \overline{X})
    \end{equation}

    % PSNR
    Szczytowy stosunek sygnału do szumu (ang. \textit{Peak Signal Noise Ratio, PSNR}). Jest udoskonaleniem błędu
    średniokwadratowego, gdyż metryka ta uzależnia swoją wartość od maksymalnej wartości przyjmowanej przez sygnał.
    Oznacza to, że taka sama wartość \textit{PSNR} odpowiada różnicom będących w proporcjonalnym do ilości informacji.
    Przykładowo, w przypadku \textit{MSE} ta sama wartość będzie przekładać się na różną postrzegalność błędu w obrazie
    korzystającym z 8 i 24 bitów na kanał. Szczytowy stosunek sygnału do szumu rozwiązuje powyższy problem. W związku z
    dużym zakresem przyjmowanych wartości metryka korzysta z skali logarytmicznej. Wartościami typowymi przy analizie
    obrazów korzystających z 8 bitów na jeden kanał jest zakres $[30dB, 50dB]$, przy czym wyższa wartość oznacza
    mniejszą degradację. Wartość metryki można wyznaczyć za pomocą wzoru \ref{eqt:psnr}.

    \begin{equation}\label{eqt:psnr}
        PSNR = 10 \cdot log_{10} \frac{MAX^2}{MSE}
    \end{equation}

    % SSIM
    Indeks podobieństwa strukturalnego (ang. \textit{Structural Similarity Index, SSIM}). Zdecydowanie bardziej złożoną
    metryką jest \textit{SSIM}. Jej celem jest uchwycenie złożoności i właściwości postrzegania ludzkiego systemu
    wzrokowego. Opiera się na założeniu mówiącym o istotności struktury obrazu w odniesieniu do postrzeganych różnic
    luminancji oraz kontrastu\cite{Wang2004ImageQA, Sara2019ImageQA}. Jej wartość jest zwykle normalizowana do zakresu
    $[0, 1]$, gdzie wartość $1$ oznacza identyczność obrazów.

    %
}

% BADANIE PERCEPCJI I OCENY OBRAZÓW
% file:///Users/grzegorzkazana/Desktop/Stegano_Ant/Comparisonoffoursubjectivemethodsforimagequalityassessment.pdf
% file:///Users/grzegorzkazana/Desktop/Stegano_Ant/ImageQualityAssessment.pdf
% file:///Users/grzegorzkazana/Desktop/Stegano_Ant/Contrast_masking_in_human_vision.pdf

% SSIM https://www.cns.nyu.edu/pub/eero/wang03-reprint.pdf
% SSIM vs PSNR https://www.semanticscholar.org/paper/Image-Quality-Metrics%3A-PSNR-vs.-SSIM-Hor%C3%A9-Ziou/4a9a98cc86b1e07e548b7edee045275a793f6698

% WYNIKI INNYCH PRAC
% - lsb/dvt
% https://www.researchgate.net/publication/329245499_LSB_Substitution_and_PVD_performance_analysis_for_image_steganography
%
% - jakaś nakoksana metoda
% file:///Users/grzegorzkazana/Desktop/Stegano_Ant/A%20novel%20algorithm%20for%20colour%20image%20steganography%20using%20a%20new%20intelligent%20technique%20based%20on%20three%20phases.pdf
%
% - wariacje IWT + sztosowe porównanie str. 24
%   file:///Users/grzegorzkazana/Desktop/Stegano_Ant/A%20Novel%20Image%20Steganographic%20Method%20based%20on%20Integer%20Wavelet%20Transformation%20and%20Particle%20Swarm%20Optimization.pdf
%
% - ACO steganography str. 6
%   file:///Users/grzegorzkazana/Desktop/Stegano_Ant/Ant%20Colony%20Optimization%20ACO%20based%20Data%20Hiding%20in%20Image%20Complex%20Region.pdf
%
% - vivbs
% file:///Users/grzegorzkazana/Desktop/Stegano_Ant/Varying_index_varying_bits_substitution_algorithm_.pdf
