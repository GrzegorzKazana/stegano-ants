\chapter{Wyniki eksperymentów}\label{chap:results}
{
    % co i po co
    %
    % jak przeprowadzano eksperymenty
    %   - na jakich obrazach (RGB, rozdzielczość)
    %   - jakie ukrywane dane (tekst)
    %   - narzucanie pojemności obrazu w celu sprawdzenia jakości
    %   - stałe parametry algorytmu (jeśli są (?))
    %
    % miary jakości i ich definicje
    %   - do czego służą
    %   - wzór
    %   - jak oddają ludzką percepcję
    %      - błąd średniokwadratowy (MSE)
    %      - szczytowy stosunek sygnału do szumu (PSNR)
    %      - podobieństwo strukturalne (SSIM)
    %
    % wyniki eksperymentów
    %   - metoda oparta o wierzchołki
    %       - omówienie parametrów algorytmu (czemu/jakie/w jakim zakresie badano)
    %       - wyniki dla różnych parametrów
    %       - krótkie wnioski
    %           - czy wpływ parametrów pokrywa się z intuicją
    %           - które systemy (dispatcher/updater) się lepiej sprawdziły, czemu
    %   - metoda oparta o krawędzie
    %       - omówienie parametrów algorytmu (czemu/jakie/w jakim zakresie badano)
    %       - omówienie metod segmentacji
    %       - wyniki dla różnych parametrów
    %       - krótkie wnioski
    %           - wpływ parametrów
    %           - wpływ ilości segmentów (krawędzi)
    %           - które systemy (dispatcher/updater) się lepiej sprawdziły, czemu
    %
    % ocena subiektywna (widzę/nie widzę)
    %   - czy pokrywa się z miarami liczbowymi
    %   - od jakiej ilości danych są widoczne
    %
    % porównanie z innymi pracami (?)
    %   - znaleźć popularne obrazy i porównać dla nich wyniki
    %   - (?) porównanie z zwykłym LSB
    %   - (?) porównanie z vLSB opartym o proste wykrywanie krawędzi

    % WSTĘP - CO TESTOWANO
    Aby zbadać skuteczność i efektywność metod zaproponowanych w rozdziale \ref{chap:stegoants} oraz podrozdziale
    \ref{sec:method}, postanowiono przeprowadzić pewne eksperymenty. Ich celem była walidacja fundamentalnych założeń,
    takich jak słuszność doboru złożonych sekcji obrazów, sprawdzenie zasadności sposobów wyznaczania reprezentacji
    grafowej problemu i interpretacji śladu feromonowego, numeryczna ocena degradacji jakości steganogramu oraz
    subiektywna opinia dotycząca postrzegalności tych zmian. Numeryczna ewaluacja degradacji jakości obrazu pozwoli
    odnieść uzyskane wyniki do istniejących rozwiązań oraz stwierdzić, czy zaproponowane metody mają swoje zastosowanie.

    % METODA BADAWCZA
    Weryfikacja polegała na przeprowadzeniu procesu ukrywania danych w bitmapach powszechnie wykorzystywanych w
    dziedzinie przetwarzania obrazów.

    % OBRAZY
    Pomimo istnienia alternatywnych zbiorów składających się z syntetycznie spreparowanych obrazów mających na celu
    wyszczególnienie pewnych cech\cite{Uhlmann2018ACI}, zdecydowano się na wykorzystanie prawdziwych zdjęć, gdyż
    bardziej odpowiadają one praktycznym zastosowaniom steganografii. W ich doborze kierowano się głównie ilością
    odniesień w tematycznie powiązanych pracach oraz bieżącymi tendencjami w kwestii wykorzystywania publicznych zbiorów
    obrazów\cite{NoteOnLena1, NoteOnLena2}. Ostatecznie, zdecydowano się skupić na obrazach \textit{Mandrill (a.k.a
    Baboon)}, \textit{Airplane (F-16) (a.k.a Jet)}, \textit{House} oraz \textit{Peppers}, które są udostępniane przez
    \textit{Uniwersytet Południowej Kalifornii (USC)}\cite{USCDatabase}.
    %% TODO: tutaj obrazki baboon, f16, peppers

    % DANE
    Dane które ukrywano w obrazach miały postać tekstu w formacie \textit{ASCII}, lecz po drobnych adaptacjach metody
    ukrywania danych w obrazie możliwe byłoby ukrywanie dowolnych danych w postaci binarnej. W eksperymentach
    wykorzystano automatycznie wygenerowany tekst \textit{Lorem ipsum} o rozmiarze $625kB$. Cechą zaproponowanej metody
    oraz zaimplementowanego programu jest możliwość skalowania wygenerowanej macierzy maskującej w taki sposób, aby
    można było umieścić zadaną ilość bitów tekstu. Pozwoli to na zbadanie degradacji jakości w zależności od objętości
    ukrywanego tekstu, oraz ułatwi porównanie rezultatów z innymi metodami.

    % BADANE PARAMETRY
    Podczas przeprowadzanych eksperymentów, badano wpływ poszczególnych metod każdego z etapów procesu oraz ich
    parametrów. Porównywane wartości parametrów dotyczą poniższych procesów.

    \begin{enumerate}
        \item Konstrukcji grafu oraz wizualizacji śladu feromonowego.
        \begin{itemize}
            \item Metoda oparta o wierzchołki. Jedynym parametrem jest opcjonalny parametr skalujący $s_0$ obraz
            wejściowy podczas budowania grafu oraz budowania macierzy maskującej. Jego wartości znajdują się w zakresie
            $[0, 1]$. Jego domyślna wartość wynosi $1$, i oznacza budowanie grafu o liczbie wierzchołków równej $w
            \times h$.

            \item Metoda oparta o krawędzie. Do jej parametrów należy algorytm segmentacji oraz docelowa liczba
            segmentów $N_s$ związana z ilością krawędzi grafu. Podczas badań wykorzystano algorytmy prostego podziału na
            na prostokąty, algorytm k-średnich oraz algorytm \textit{SLIC} służący do konstrukcji superpikseli.
        \end{itemize}

        \item Wyznaczenie śladu feromonowego przez różne rodzaje systemów mrówkowych. Do parametrów wspólnych dla
        każdego z wykorzystanych systemów należy:
        \begin{itemize}
            \item ilość mrówek $A$, która domyślnie jest równa ilości wierzchołków grafu $V$,
            \item liczba wykonanych cykli $C$
            \item liczba kroków wykonywanych przez mrówki w każdej iteracji algorytmu $S$, dla grafów skonstruowanych na
            podstawie segmentacji obrazu wynosi jest ona równa ilości wierzchołków $V$
            \item preferencja względem śladu feromonowego $\alpha$
            \item preferencja względem widoczności wierzchołka $\beta$
            \item początkowe natężenie śladu feromonowego $\tau_0$
            \item współczynnik opisujący prędkość wyparowania śladu feromonowego $\rho$
        \end{itemize}

        Do porównanych rodzajów systemów należą poniższe odmiany. Z niektórymi z nich związane są dodatkowe parametry
        oraz domyślne wartości powyższych atrybutów.
        \begin{itemize}
            \item model feromon stały,
            \item model feromon średni,
            \item model feromon cykliczny,
            \item system mrowiskowy, który wprowadza prawdopodobieństwo eksploatacji najkrótszej krawędzi $q_0$ oraz
            ustalona parametr $\alpha = 1$,
            \item system max-min, który wprowadza ograniczenia wartości śladu feromonowego $[\tau_{min}, \tau_{max}]$.
            Ich wartości wyznaczane są za pomocą estymaty dotyczącej długości poszukiwanego cyklu. Wartość początkowa
            feromonu wynosi $\tau_0 = \tau_{max}$.
        \end{itemize}
    \end{enumerate}

    % WYKORZYSTANE MIARY JAKOŚCI
    W celu umożliwienia porównywania jakości steganogramów, zdecydowano się skorzystać z metryk służących
    do porównawczej analizy obrazów. Zastosowane metryki można podzielić na dwie kategorie, obiektywne i subiektywne.
    Metryki obiektywne służą do wyznaczenia pewnej wartości charakteryzującej różnicę pomiędzy dwoma sygnałami. W
    przypadku metryk subiektywnych, ich zadaniem jest również wyznaczenie wartości numerycznej, lecz nacisk kładziony
    jest na korelację wartości z postrzegalnością zmian przez ludzki układ wzrokowy. Do zastosowanych metryk należą:

    % MSE
    Błąd średniokwadratowy (ang. \textit{Mean Square Error, MSE}). Jest jedną z najprostszych metryk służących do
    pomiaru różnic między obrazami. Jej wartości należą do zbioru nieujemnych liczb rzeczywistych, a wartości bliższe
    zeru stanowią o mniejszym spadku jakości. Do jej zalet należy prostota implementacji i możliwość optymalnej
    implementacji. Jedną z jej wad jest niska korelacja z postrzeganiem różnic między obrazami przez ludzi oraz
    nieuwzględnianie informacji o relacji natężenia szumu do wartości sygnału. Jej wartość wyraża wzór \ref{eqt:mse}, w
    którym $\overline{X}$ oznacza wartość średnią.

    \begin{equation}\label{eqt:mse}
        MSE = \frac{1}{n} \sum_{i=1}^n (X_i - \overline{X})
    \end{equation}

    % PSNR
    Szczytowy stosunek sygnału do szumu (ang. \textit{Peak Signal Noise Ratio, PSNR}). Jest udoskonaleniem błędu
    średniokwadratowego, gdyż metryka ta uzależnia swoją wartość od maksymalnej wartości przyjmowanej przez sygnał.
    Oznacza to, że taka sama wartość \textit{PSNR} odpowiada różnicom będących w proporcjonalnym do ilości informacji.
    Przykładowo, w przypadku \textit{MSE} ta sama wartość będzie przekładać się na różną postrzegalność błędu w obrazie
    korzystającym z 8 i 24 bitów na kanał. Szczytowy stosunek sygnału do szumu rozwiązuje powyższy problem. W związku z
    dużym zakresem przyjmowanych wartości metryka korzysta z skali logarytmicznej. Wartościami typowymi przy analizie
    obrazów korzystających z 8 bitów na jeden kanał jest zakres $[30dB, 50dB]$, przy czym wyższa wartość oznacza
    mniejszą degradację. Wartość metryki można wyznaczyć za pomocą wzoru \ref{eqt:psnr}.

    \begin{equation}\label{eqt:psnr}
        PSNR = 10 \cdot log_{10} \frac{MAX^2}{MSE}
    \end{equation}

    % SSIM
    Indeks podobieństwa strukturalnego (ang. \textit{Structural Similarity Index, SSIM}). Zdecydowanie bardziej złożoną
    metryką jest \textit{SSIM}. Jej celem jest uchwycenie złożoności i właściwości postrzegania ludzkiego systemu
    wzrokowego. Opiera się na założeniu mówiącym o istotności struktury obrazu w odniesieniu do postrzeganych różnic
    luminancji oraz kontrastu\cite{Wang2004ImageQA, Sara2019ImageQA}. Jej wartość jest zwykle normalizowana do zakresu
    $[0, 1]$, gdzie wartość $1$ oznacza identyczność obrazów.

    % WYNIKI EKSPERYMENTÓW
    Eksperymenty rozpoczęto od analizy wyników uzyskanych metodą opartą o budowanie grafu na podstawie wierzchołków.

    % WYNIKI - METODA-WIERZCHOŁKÓW - INVERTED
    Eksperymenty rozpoczęto od porównania wpływu proporcjonalności długości krawędzie grafu od różnicy pomiędzy
    pikselami. Jak wspomniano w rozdziale \ref{sec:method}, jeśli długości krawędzi są proporcjonalne do różnicy
    pikseli, system mrówkowy będzie dążył do odkładania śladu feromonowego w obszarach cechujących mniejszą złożoność.
    Aby uzyskać zamierzony rezultat, czyli macierz maskującą wskazującą obszary bardziej złożone, konieczne będzie
    odwrócenie jej wartości. Alternatywnie, jeśli długości pomiędzy pikselami będą odwrotnie proporcjonalne do dzielącej
    je odległości, system mrówkowy, a co za tym idzie ślad feromonowy, będzie preferował obszary bardziej złożone.

    Wizualizację procesu konwersji obrazu na graf oraz uzyskane macierze maskujące przedstawia rysunek
    \ref{fig:exp-vertex}. W pierwszej kolumnie umieszczono obrazy objaśniające wynik konwersji bitmapy na graf, w
    których do poszczególnych pikseli obrazu wejściowego przypisana jest luminancja odwrotnie proporcjonalna do długości
    krawędzi z nim związanych (obszary ,,atrakcyjniejsze'' dla mrówek cechują się jaśniejszym kolorem). Kolumna druga
    przedstawia macierze maskujące, które będą decydować o ilości bitów które zostaną zamienione na bity ukrywanej
    wiadomości. Im jaśniejszy jest dany piksel, tym więcej jego bitów zostanie zamienione. W trzeciej kolumnie
    przedstawiono macierze maskujące przetransformowane w taki sposób, aby odpowiadające im pojemność ukrywanej
    wiadomości była taka sama.

    Na podstawie obrazów \ref{fig:exp-vertex-pher} oraz \ref{fig:exp-vertex-inv-pher} można stwierdzić, że obie
    metody poprawnie wykrywają krawędzie i obszary złożone, lecz po porównaniu macierzy maskujących o tej samej
    pojemności należy stwierdzić lepszy rezultat procesu w którym długości krawędzi grafu są proporcjonalne do różnicy
    pomiędzy pikselami, a macierz maskująca zostaje odwrócona. Przeskalowana macierz maskująca
    \ref{fig:exp-vertex-inv-pher-scaled} gwarantuje bardziej równomierne rozłożenie informacji przy jednoczesnym
    faworyzowaniu krawędzi o obszarów złożonych.

    \begin{figure}
        \centering
        \subfloat[][]{
            \label{fig:exp-vertex-conv}
            \includegraphics[width=4cm]{experiments/vertexbased/normal-vs-inverted/2021-04-17-06:25:53house-l_conv_a1000_s100_Dbasic:_Uconst:_Cspatial:_c10_m0_t50000B_}

        }
        \hspace{8pt}
        \subfloat[][]{
            \label{fig:exp-vertex-pher}
            \includegraphics[width=4cm]{experiments/vertexbased/normal-vs-inverted/2021-04-17-06:25:53house-l_pher_a1000_s100_Dbasic:_Uconst:_Cspatial:_c10_m0_t50000B_}
        }
        \hspace{8pt}
        \subfloat[][]{
            \label{fig:exp-vertex-pher-scaled}
            \includegraphics[width=4cm]{experiments/vertexbased/normal-vs-inverted/2021-04-17-06:25:53house-l_pher_scaled_a1000_s100_Dbasic:_Uconst:_Cspatial:_c10_m0_t50000B_}
        }
        \\
        \subfloat[][]{
            \label{fig:exp-vertex-inv-conv}
            \includegraphics[width=4cm]{experiments/vertexbased/normal-vs-inverted/2021-04-17-06:26:02house-l_conv_a1000_s100_Dbasic:_Uconst:_Ci:spatial:_c10_m0_t50000B_}
        }
        \hspace{8pt}
        \subfloat[][]{
            \label{fig:exp-vertex-inv-pher}
            \includegraphics[width=4cm]{experiments/vertexbased/normal-vs-inverted/2021-04-17-06:26:02house-l_pher_a1000_s100_Dbasic:_Uconst:_Ci:spatial:_c10_m0_t50000B_}
        }
        \subfloat[][]{
            \label{fig:exp-vertex-inv-pher-scaled}
            \includegraphics[width=4cm]{experiments/vertexbased/normal-vs-inverted/2021-04-17-06:26:02house-l_pher_scaled_a1000_s100_Dbasic:_Uconst:_Ci:spatial:_c10_m0_t50000B_}
        }

        \caption[Porównania wizualizacji konwersji oraz macierzy maskujących.]
        {
            Porównania wizualizacji konwersji oraz macierzy maskujących.
            W pierwszym wierszu (\subref{fig:exp-vertex-conv}, \subref{fig:exp-vertex-pher}) przedstawiono obrazy
            związane z procesem o długości krawędzie odwrotnie proporcjonalnej do różnicy pikseli. W drugim wierszu
            (\subref{fig:exp-vertex-inv-conv}, \subref{fig:exp-vertex-inv-pher}) długości są proporcjonalne do różnicy
            pikseli. Pierwsza kolumna przedstawia wizualizacje procesu konwersji bitmapy na graf, druga przestawia
            uzyskane macierze maskujące, trzecia zawiera macierze maskujące przeskalowane w taki sposób, aby odpowiadała
            im taka sama pojemność ukrywanej informacji.
        }
        \label{fig:exp-vertex}
    \end{figure}

    % % WYNIKI - METODA-WIERZCHOŁKÓW - C | S | A
    % Następnie, porównano wpływ ilości iteracji algorytmu (ilość cykli $C$), ilośc kroków wykonanych przez mrówki $S$
    % w każdym cyklu, oraz ilość mrówek. Ponieważ każdy z powyższych parametrów

    % WYNIKI - METODA-WIERZCHOŁKÓW - RODZAJE SYSTEMÓW
    Kolejnym parametrem który postanowiono zbadać, jest rodzaj systemu mrówkowego oraz związane z nim parametry.
    Ponieważ problem rozwiązywany przez mrówki nie jest równoważny z problemem komiwojażera, konieczne jest również
    ustalenie parametru $S$ określającego ilość kroków wykonywanych przez każdą mrówkę w każdej iteracji algorytmu.

    Wygenerowane macierze maskujące przedstawia rysunek \ref{fig:exp-vertex-pher}. Na ich podstawie można wyciągnąć
    kilka interesujących wniosków. System mrowiskowy (\textit{Ant-Colony}) oraz system \textit{Max-min} nie odpowiadają
    charakterystyce zadanego zadania, które jest znacząco różne od klasycznego problemu \textit{TSP}. Przyczyn można
    doszukiwać się w silnej faworyzacji najlepszego rozwiązania, czyli najkrótszej obranej ścieżki kosztem pozostałych
    ścieżek. Można wnioskować, że w przedstawionym problemie znalezienie najkrótszej ścieżki łączącej $S + 1$ pikseli
    spośród $w \times h$ pikseli nie przyczynia się do ogólnego wyznaczenia obszarów obrazu.

    Na dodatkową uwagę zasługuje system z cykliczną reguła aktualizacji śladu feromonowego, gdyż za jego pomocą udało
    się wyraźnie wyodrębnić jednorodny obszar nieba oraz pewne homogeniczne elementy budynku. Kolejną obserwacją
    płynącą z powyższego eksperymentu jest istotność doboru ilości mrówek $A$ oraz współczynnika związanego z
    odparowaniem śladu $\rho$. Macierze lepiej odzwierciedlające faktyczne segmenty obrazów powstały w przypadku
    konfiguracji charakteryzujących się większą ilością mrówek (na przykład \ref{fig:exp-vertex-pher-pher-cycle3}) lub
    wyższym współczynnikiem $\rho$ (na przykład \ref{fig:exp-vertex-pher-conv-const1}), przekładającym się na
    wolniejsze tempo odparowania.

    \begin{figure}
        \footnotesize
        \centering
        \subfloat[][Ant-Density\newline $A=1000$, $S=100$ \newline $\alpha=2$, $\beta=1$, $\rho=0.8$]{
            \label{fig:exp-vertex-pher-conv-const1}
            \includegraphics[width=3cm]{experiments/vertexbased/colony-types/2021-04-17-12:29:22house-l_pher_a1000_s100_Dbiased:2,1_Uconst:1,0.2,1_Ci:spatial:_c10_m0_t100000B_}
        }
        \hspace{8pt}
        \subfloat[][Ant-Density\newline $A=1000$, $S=100$ \newline $\alpha=2$, $\beta=2$, $\rho=0.9$]{
            \label{fig:exp-vertex-pher-pher-const2}
            \includegraphics[width=3cm]{experiments/vertexbased/colony-types/2021-04-17-12:29:25house-l_pher_a1000_s100_Dbiased:2,2_Uconst:1,0.1,1_Ci:spatial:_c10_m0_t100000B_}
        }
        \hspace{8pt}
        \subfloat[][Ant-Density\newline $A=1000$, $S=100$ \newline $\alpha=2$, $\beta=2$, $\rho=0.999$]{
            \label{fig:exp-vertex-pher-pher-const3}
            \includegraphics[width=3cm]{experiments/vertexbased/colony-types/2021-04-17-12:29:25house-l_pher_a1000_s100_Dbiased:2,2_Uconst:1,0.001,1_Ci:spatial:_c10_m0_t100000B_}
        }
        \\
        \subfloat[][Ant-Quantity\newline $A=1000$, $S=100$ \newline $\alpha=1$, $\beta=1$, $\rho=0.8$]{
            \label{fig:exp-vertex-pher-pher-avg1}
            \includegraphics[width=3cm]{experiments/vertexbased/colony-types/2021-04-17-12:29:31house-l_pher_a1000_s100_Dbiased:1,1_Uavg:1,0.2,1_Ci:spatial:_c10_m0_t100000B_}
        }
        \hspace{8pt}
        \subfloat[][Ant-Quantity\newline $A=1000$, $S=100$ \newline $\alpha=2$, $\beta=2$, $\rho=0.999$]{
            \label{fig:exp-vertex-pher-pher-avg2}
            \includegraphics[width=3cm]{experiments/vertexbased/colony-types/2021-04-17-12:29:29house-l_pher_a1000_s100_Dbiased:2,2_Uavg:1,0.001,1_Ci:spatial:_c10_m0_t100000B_}
        }
        \subfloat[][Ant-Quantity\newline $A=10000$, $S=10$ \newline $\alpha=1$, $\beta=1$, $\rho=0.8$]{
            \label{fig:exp-vertex-pher-pher-avg3}
            \includegraphics[width=3cm]{experiments/vertexbased/colony-types/2021-04-17-12:29:12house-l_pher_a10000_s10_Dbiased:1,1_Uavg:1,0.2,1_Ci:spatial:_c10_m0_t100000B_}
        }
        \\
        \subfloat[][Ant-Cycle\newline $A=10$, $S=10000$ \newline $\alpha=1$, $\beta=1$, $\rho=0.8$]{
            \label{fig:exp-vertex-pher-pher-cycle1}
            \includegraphics[width=3cm]{experiments/vertexbased/colony-types/2021-04-17-12:29:38house-l_pher_a10_s10000_Dbiased:1,1_Ucycle:1,0.2,1,10000_Ci:spatial:_c10_m0_t100000B_}
        }
        \hspace{8pt}
        \subfloat[][Ant-Cycle\newline $A=1000$, $S=100$ \newline $\alpha=1$, $\beta=1$, $\rho=0.8$]{
            \label{fig:exp-vertex-pher-pher-cycle2}
            \includegraphics[width=3cm]{experiments/vertexbased/colony-types/2021-04-17-12:29:26house-l_pher_a1000_s100_Dbiased:1,1_Ucycle:1,0.2,1,100_Ci:spatial:_c10_m0_t100000B_}
        }
        \subfloat[][Ant-Cycle\newline $A=10000$, $S=10$ \newline $\alpha=1$, $\beta=1$, $\rho=0.8$]{

            \label{fig:exp-vertex-pher-pher-cycle3}
            \includegraphics[width=3cm]{experiments/vertexbased/colony-types/2021-04-17-12:29:10house-l_pher_a10000_s10_Dbiased:1,1_Ucycle:1,0.2,1,10_Ci:spatial:_c10_m0_t100000B_}
        }
        \\
        \subfloat[][Ant-Colony\newline $A=10$, $S=10000$ \newline $\alpha=1$, $\beta=1$, $\rho=0.999$]{
            \label{fig:exp-vertex-pher-pher-colony1}
            \includegraphics[width=3cm]{experiments/vertexbased/colony-types/2021-04-17-12:35:20house-l_pher_a10_s10000_Dcolony:0.1,1_Ucolony:1,0.2,0.2,10000_Ci:spatial:_c10_m0_t100000B_}
        }
        \hspace{8pt}
        \subfloat[][Ant-Colony\newline $A=1000$, $S=100$ \newline $\alpha=1$, $\beta=2$, $\rho=0.999$]{
            \label{fig:exp-vertex-pher-pher-colony2}
            \includegraphics[width=3cm]{experiments/vertexbased/colony-types/2021-04-17-12:29:29house-l_pher_a1000_s100_Dcolony:0.1,2_Ucolony:1,0.001,0.001,100_Ci:spatial:_c10_m0_t100000B_}
        }
        \subfloat[][Ant-Colony\newline $A=10000$, $S=10$ \newline $\alpha=1$, $\beta=1$, $\rho=0.8$]{
            \label{fig:exp-vertex-pher-pher-colony3}
            \includegraphics[width=3cm]{experiments/vertexbased/colony-types/2021-04-17-12:29:11house-l_pher_a10000_s10_Dcolony:0.1,1_Ucolony:1,0.2,0.2,10_Ci:spatial:_c10_m0_t100000B_}
        }
        \\
        \subfloat[][Max-Min\newline $A=10$, $S=10000$ \newline $\alpha=1$, $\beta=1$, $\rho=0.8$]{
            \label{fig:exp-vertex-pher-pher-maxmin1}
            \includegraphics[width=3cm]{experiments/vertexbased/colony-types/2021-04-17-12:29:38house-l_pher_a10_s10000_Dbiased:1,1_Umaxmin:1,0.2,0.1,10000_Ci:spatial:_c10_m0_t100000B_}
        }
        \hspace{8pt}
        \subfloat[][Max-Min\newline $A=1000$, $S=100$ \newline $\alpha=2$, $\beta=2$, $\rho=0.999$]{
            \label{fig:exp-vertex-pher-pher-maxmin2}
            \includegraphics[width=3cm]{experiments/vertexbased/colony-types/2021-04-17-12:29:27house-l_pher_a1000_s100_Dbiased:2,2_Umaxmin:1,0.001,0.1,100_Ci:spatial:_c10_m0_t100000B_}
        }
        \subfloat[][Max-Min\newline $A=10000$, $S=10$ \newline $\alpha=1$, $\beta=1$, $\rho=0.9$]{
            \label{fig:exp-vertex-pher-pher-maxmin3}
            \includegraphics[width=3cm]{experiments/vertexbased/colony-types/2021-04-17-12:29:12house-l_pher_a10000_s10_Dbiased:1,1_Umaxmin:1,0.2,0.1,10_Ci:spatial:_c10_m0_t100000B_}
        }

        \caption[Porównania wizualizacji konwersji oraz macierzy maskujących.]
        {
            Macierze maskujące wygenerowane przy pomocy różnych systemów mrówkowych.
        }
        \label{fig:exp-vertex-pher}
    \end{figure}

    % METODA WIERZCHOŁKÓW - SKALOWANIE OBRAZU
    Ponieważ pozostałe obrazy z zbioru \textit{USC} mają rozmiary $512 \times 512$, zdecydowano się również zbadać wpływ
    skalowania obrazu podczas tworzenia grafu oraz macierzy maskującej. Wykorzystano system oparty o model cykliczny o
    parametrach $A=10000$, $S=100$, $\alpha=\beta=1$, $\rho=0.8$, i starano się umieścić $250 000kB$ danych. Na rysunku
    \ref{fig:exp-vertex-scale} zamieszczono macierze maskujące powstałe przy $s_0 \in \{0.25, 0.5, 0.75, 1.0\}$. Tabela
    \ref{tab:exp-vertex-scale-errors} zawiera porównanie miar jakości steganogramów.

    Na postawie uzyskanych wyników można stwierdzić że tworzenie grafu na podstawie skalowanego obrazu nie ma znaczącego
    wpływu na miary jakości steganogramu. Jest to pożądana cecha, ponieważ operacje na mniejszych grafach zajmują
    zdecydowanie mniej czasu.

    \begin{figure}
        \footnotesize
        \centering
        \subfloat[][Macierz maskująca\newline $s_0 = 0.25$]{
            \label{fig:exp-vertex-scale0.25-pher}
            \includegraphics[width=4cm]{experiments/vertexbased/scaling/2021-04-17-15:53:35airplane_pher_a10000_s100_Dbiased:1,1_Ucycle:1,0.2,1,100_Ci:spatial:_c10_m128_t250000B_}
        }
        \hspace{8pt}
        \subfloat[][Steganogram\newline $s_0 = 0.25$]{
            \label{fig:exp-vertex-scale0.25-steg}
            \includegraphics[width=4cm]{experiments/vertexbased/scaling/2021-04-17-15:53:35airplane_steg_a10000_s100_Dbiased:1,1_Ucycle:1,0.2,1,100_Ci:spatial:_c10_m128_t250000B_}
        }
        \\
        \subfloat[][Macierz maskująca\newline $s_0 = 0.5$]{
            \label{fig:exp-vertex-scale0.50-pher}
            \includegraphics[width=4cm]{experiments/vertexbased/scaling/2021-04-17-15:53:51airplane_pher_a10000_s100_Dbiased:1,1_Ucycle:1,0.2,1,100_Ci:spatial:_c10_m256_t250000B_}
        }
        \hspace{8pt}
        \subfloat[][Steganogram\newline $s_0 = 0.5$]{
            \label{fig:exp-vertex-scale0.50-steg}
            \includegraphics[width=4cm]{experiments/vertexbased/scaling/2021-04-17-15:53:51airplane_steg_a10000_s100_Dbiased:1,1_Ucycle:1,0.2,1,100_Ci:spatial:_c10_m256_t250000B_}
        }
        \\
        \subfloat[][Macierz maskująca\newline $s_0 = 0.75$]{
            \label{fig:exp-vertex-scale0.75-pher}
            \includegraphics[width=4cm]{experiments/vertexbased/scaling/2021-04-17-15:56:42airplane_pher_a10000_s100_Dbiased:1,1_Ucycle:1,0.2,1,100_Ci:spatial:_c10_m384_t250000B_}
        }
        \hspace{8pt}
        \subfloat[][Steganogram\newline $s_0 = 0.75$]{

            \label{fig:exp-vertex-scale0.75-steg}
            \includegraphics[width=4cm]{experiments/vertexbased/scaling/2021-04-17-15:56:43airplane_steg_a10000_s100_Dbiased:1,1_Ucycle:1,0.2,1,100_Ci:spatial:_c10_m384_t250000B_}
        }
        \\
        \subfloat[][Macierz maskująca\newline $s_0 = 1.0$]{
            \label{fig:exp-vertex-scale1.0-pher}
            \includegraphics[width=4cm]{experiments/vertexbased/scaling/2021-04-17-15:54:02airplane_pher_a10000_s100_Dbiased:1,1_Ucycle:1,0.2,1,100_Ci:spatial:_c10_m0_t250000B_}
        }
        \hspace{8pt}
        \subfloat[][Steganogram\newline $s_0 = 1.0$]{
            \label{fig:exp-vertex-scale1.0-steg}
            \includegraphics[width=4cm]{experiments/vertexbased/scaling/2021-04-17-15:54:02airplane_steg_a10000_s100_Dbiased:1,1_Ucycle:1,0.2,1,100_Ci:spatial:_c10_m0_t250000B_}
        }

        \caption[Porównania wizualizacji konwersji oraz macierzy maskujących.]
        {
            Macierze maskujące oraz steganogramy wygenerowane dla różnych wartości skali $s_0$.
        }
        \label{fig:exp-vertex-scale}
    \end{figure}

    \begin{center}
        \begin{tabular}{ |c|c c c| }
            \hline
            Współczynnik skalujący \newline $s_0$ & $MSE$ & $PSNR$ & $SSIM$ \\
            \hline
            0.25 & 6.93092 & 39.7568dB & \textbf{0.952935} \\
            0.5 & 6.67252 & 39.9218dB & 0.951723 \\
            0.75 & 6.62924 & 39.9501dB & 0.95122 \\
            1.0 & \textbf{6.61124} & \textbf{39.9619dB} & 0.948736 \\
            \hline
        \end{tabular}
        \label{tab:exp-vertex-scale-errors}
    \end{center}

    % METODA WIERZCHOŁKÓW - FINAL SHOWDOWN
    Ostatecznie, postanowiono zbadać wyniki uzyskane na pozostałych obrazach testowych. Uzyskane rezultaty zawiera
    tabela \ref{tab:exp-vertex-results} a steganogramy rysunek \ref{fig:exp-vertex-results}. Parametry algorytmu
    zachowano z poprzedniego eksperymentu.

    \begin{center}
        \begin{tabular}{ |c|c|c c c| }
            \hline
            Obraz & objętość danych & $MSE$ & $PSNR$ & $SSIM$ \\
            \hline
            Baboon {\footnotesize $512 \times 512$}   & 100 000B (12.71\%) & 0.538851 & 50.85dB & 0.998768 \\
            Baboon {\footnotesize $512 \times 512$}   & 250 000B (31.78\%) & 6.60594 & 39.9654dB & 0.990015 \\
            Baboon {\footnotesize $512 \times 512$}   & 500 000B (63.57\%) & 272.732 & 23.8074dB & 0.777025 \\
            Airplane {\footnotesize $512 \times 512$} & 100 000B (12.71\%) & 0.53689 & 50.8659dB & 0.995503 \\
            Airplane {\footnotesize $512 \times 512$} & 250 000B (31.78\%) & 6.67252 & 39.9218dB & 0.951723 \\
            Airplane {\footnotesize $512 \times 512$} & 500 000B (63.57\%) & 311.43 & 23.2311dB & 0.516503 \\
            Peppers {\footnotesize $512 \times 512$}  & 100 000B (12.71\%) & 0.536934 & 50.8655dB & 0.996048 \\
            Peppers {\footnotesize $512 \times 512$}  & 250 000B (31.78\%) & 6.71828 & 39.8922dB & 0.958181 \\
            Peppers {\footnotesize $512 \times 512$}  & 500 000B (63.57\%) & 292.99 & 23.4962dB & 0.502009 \\
            \hline
        \end{tabular}
        \label{tab:exp-vertex-results}
    \end{center}

    \begin{figure}
        \footnotesize
        \centering
        \subfloat[][$D = 100 000B$]{
            \label{fig:exp-vertex-results-baboon-100k}
            \includegraphics[width=3cm]{experiments/vertexbased/results/2021-04-17-17:12:31mandrill_steg_a10000_s100_Dbiased:1,1_Ucycle:1,0.2,1,100_Ci:spatial:_c10_m256_t100000B_}
        }
        \hspace{8pt}
        \subfloat[][$D = 250 000B$]{
            \label{fig:exp-vertex-results-baboon-250k}
            \includegraphics[width=3cm]{experiments/vertexbased/results/2021-04-17-17:13:06mandrill_steg_a10000_s100_Dbiased:1,1_Ucycle:1,0.2,1,100_Ci:spatial:_c10_m256_t250000B_}
        }
        \subfloat[][$D = 500 000B$]{
            \label{fig:exp-vertex-results-baboon-500k}
            \includegraphics[width=3cm]{experiments/vertexbased/results/2021-04-17-17:13:33mandrill_steg_a10000_s100_Dbiased:1,1_Ucycle:1,0.2,1,100_Ci:spatial:_c10_m256_t500000B_}
        }
        \\
        \subfloat[][$D = 100 000B$]{
            \label{fig:exp-vertex-results-aiplane-100k}
            \includegraphics[width=3cm]{experiments/vertexbased/results/2021-04-17-17:16:39airplane_steg_a10000_s100_Dbiased:1,1_Ucycle:1,0.2,1,100_Ci:spatial:_c10_m256_t100000B_}
        }
        \hspace{8pt}
        \subfloat[][$D = 250 000B$]{
            \label{fig:exp-vertex-results-aiplane-250k}
            \includegraphics[width=3cm]{experiments/vertexbased/results/2021-04-17-17:17:10airplane_steg_a10000_s100_Dbiased:1,1_Ucycle:1,0.2,1,100_Ci:spatial:_c10_m256_t250000B_}
        }
        \hspace{8pt}
        \subfloat[][$D = 500 000B$]{
            \label{fig:exp-vertex-results-aiplane-500k}
            \includegraphics[width=3cm]{experiments/vertexbased/results/2021-04-17-17:17:35airplane_steg_a10000_s100_Dbiased:1,1_Ucycle:1,0.2,1,100_Ci:spatial:_c10_m256_t500000B_}
        }
        \\
        \subfloat[][$D = 100 000B$]{
            \label{fig:exp-vertex-results-airplane-100k}
            \includegraphics[width=3cm]{experiments/vertexbased/results/2021-04-17-17:18:34peppers_steg_a10000_s100_Dbiased:1,1_Ucycle:1,0.2,1,100_Ci:spatial:_c10_m256_t100000B_}
        }
        \hspace{8pt}
        \subfloat[][$D = 250 000B$]{
            \label{fig:exp-vertex-results-airplane-250k}
            \includegraphics[width=3cm]{experiments/vertexbased/results/2021-04-17-17:19:05peppers_steg_a10000_s100_Dbiased:1,1_Ucycle:1,0.2,1,100_Ci:spatial:_c10_m256_t250000B_}
        }
        \subfloat[][$D = 500 000B$]{
            \label{fig:exp-vertex-results-airplane-500k}
            \includegraphics[width=3cm]{experiments/vertexbased/results/2021-04-17-17:19:31peppers_steg_a10000_s100_Dbiased:1,1_Ucycle:1,0.2,1,100_Ci:spatial:_c10_m256_t500000B_}
        }

        \caption[Porównanie rezultatów]
        {
            Porównanie uzyskanych steganogramów przy zadanej objętości ukrytej informacji.
        }
        \label{fig:exp-vertex-results}
    \end{figure}

    % Dla każdego z rodzaju systemów przeprowadzono TODO: N kombinacji parametrów oraz wykonano 10 cykli. Wyniki
    % zamieszczono w tabeli TODO: ref, a najlepsze macierze maskujące dla każdego systemu przedstawia rysunek TODO:. W
    % każdym przypadku w macierz maskująca została przeskalowana w taki sposób, aby w steganogramie ukryć $50kB$.
    % Eksperymenty przeprowadzono na przykładzie obrazu \textit{House} o wymiarach $200 \times 200$.
    % \begin{center}
    %     \footnotesize
    %     \begin{tabular}{ |l|c c c c c c|c c c| }
    %         \hline
    %         Rodzaj systemu & $A$ & $S$ & $\alpha$ & $\beta$ & $\rho_0$ & $q_0$ & $MSE$ & $PSNR$ & $SSIM$ \\
    %         \hline
    %         feromon stały & 10 & 10000 & 1 & 1 & 0.8 & - & 2347.05 & 14.4595dB & 0.163352 \\
    %         feromon stały & 10000 & 10 & 1 & 1 & 0.8 & - & 1829.54 & 15.5413dB & 0.197731 \\
    %         feromon stały & 1000 & 100 & 1 & 1 & 0.8 & - & 1878.97 & 15.4255dB & 0.215501 \\
    %         feromon stały & 1000 & 100 & 1 & 2 & 0.8 & - & 1873.37 & 15.4385dB & 0.217721 \\
    %         feromon stały & 1000 & 100 & 2 & 1 & 0.8 & - & 1867.24 & 15.4527dB & 0.219012 \\
    %         feromon stały & 1000 & 100 & 2 & 2 & 0.8 & - & 1870.21 & 15.4458dB & 0.21761 \\
    %         feromon stały & 1000 & 100 & 2 & 2 & 0.5 & - & 1860.48 & 15.4685dB & 0.217551 \\
    %         feromon stały & 1000 & 100 & 2 & 2 & 0.9 & - & 1864.67 & 15.4587dB & 0.214871 \\
    %         feromon stały & 1000 & 100 & 2 & 2 & 0.999 & - & 1842.56 & 15.5105dB & 0.196247 \\
    %         feromon średni & 10 & 10000 & 1 & 1 & 0.8 & - & 2347.05 & 14.4595dB & 0.163352 \\
    %         feromon średni & 10000 & 10 & 1 & 1 & 0.8 & - & 1859.78 & 15.4701dB & 0.218718 \\
    %         feromon średni & 1000 & 100 & 1 & 1 & 0.8 & - & 1874.76 & 15.4353dB & 0.217647 \\
    %         feromon średni & 1000 & 100 & 1 & 2 & 0.8 & - & 1863.55 & 15.4613dB & 0.219927 \\
    %         feromon średni & 1000 & 100 & 2 & 1 & 0.8 & - & 1866.55 & 15.4543dB & 0.221946 \\
    %         feromon średni & 1000 & 100 & 2 & 2 & 0.8 & - & 1871.49 & 15.4429dB & 0.223166 \\
    %         feromon średni & 1000 & 100 & 2 & 2 & 0.5 & - & 1864.72 & 15.4586dB & 0.224912 \\
    %         feromon średni & 1000 & 100 & 2 & 2 & 0.9 & - & 1857.8 & 15.4747dB & 0.221431 \\
    %         feromon średni & 1000 & 100 & 2 & 2 & 0.999 & - & 1863.9 & 15.4605dB & 0.220019 \\
    %         feromon cykliczny & 10 & 10000 & 1 & 1 & 0.8 & - & 1848.07 & 15.4975dB & 0.218536 \\
    %         feromon cykliczny & 10000 & 10 & 1 & 1 & 0.8 & - & 1901.89 & 15.3729dB & 0.196128 \\
    %         feromon cykliczny & 1000 & 100 & 1 & 1 & 0.8 & - & 1924.33 & 15.3219dB & 0.18948 \\
    %         feromon cykliczny & 1000 & 100 & 1 & 2 & 0.8 & - & 1886.63 & 15.4079dB & 0.208189 \\
    %         feromon cykliczny & 1000 & 100 & 2 & 1 & 0.8 & - & 1929.84 & 15.3095dB & 0.19839 \\
    %         feromon cykliczny & 1000 & 100 & 2 & 2 & 0.8 & - & 1887.68 & 15.4055dB & 0.210202 \\
    %         feromon cykliczny & 1000 & 100 & 2 & 2 & 0.5 & - & 1902.63 & 15.3712dB & 0.206376 \\
    %         feromon cykliczny & 1000 & 100 & 2 & 2 & 0.9 & - & 1906.93 & 15.3614dB & 0.202576 \\
    %         feromon cykliczny & 1000 & 100 & 2 & 2 & 0.999 & - & 1888.45 & 15.4037dB & 0.209342 \\
    %         sys. mrowiskowy & 10 & 10000 & 1 & 1 & 0.8 & 0.1 & 1863.02 & 15.4626dB & 0.220336 \\
    %         sys. mrowiskowy & 10000 & 10 & 1 & 1 & 0.8 & 0.1 & 1860.85 & 15.4676dB & 0.227484 \\
    %         sys. mrowiskowy & 1000 & 100 & 1 & 1 & 0.8 & 0.1 & 1862.1 & 15.4647dB & 0.227632 \\
    %         sys. mrowiskowy & 1000 & 100 & 1 & 1 & 0.8 & 0.1 & 1862.1 & 15.4647dB & 0.227632 \\
    %         sys. mrowiskowy & 1000 & 100 & 1 & 2 & 0.8 & 0.1 & 1862.86 & 15.4629dB & 0.224279 \\
    %         sys. mrowiskowy & 1000 & 100 & 1 & 2 & 0.8 & 0.1 & 1862.86 & 15.4629dB & 0.224279 \\
    %         sys. mrowiskowy & 1000 & 100 & 1 & 2 & 0.8 & 0.1 & 1862.86 & 15.4629dB & 0.224279 \\
    %         sys. mrowiskowy & 1000 & 100 & 1 & 2 & 0.5 & 0.1 & 1877.58 & 15.4288dB & 0.219284 \\
    %         sys. mrowiskowy & 1000 & 100 & 1 & 2 & 0.9 & 0.1 & 1873.54 & 15.4381dB & 0.218873 \\
    %         sys. mrowiskowy & 1000 & 100 & 1 & 2 & 0.999 & 0.1 & 1513.26 & 16.3656dB & 0.231456 \\
    %         max-min & 10 & 10000 & 1 & 1 & 0.8 & - & 1742.49 & 15.753dB & 0.192021 \\
    %         max-min & 10000 & 10 & 1 & 1 & 0.8 & - & 1352.97 & 16.8519dB & 0.260953 \\
    %         max-min & 1000 & 100 & 1 & 1 & 0.8 & - & 1864.2 & 15.4598dB & 0.224875 \\
    %         max-min & 1000 & 100 & 1 & 2 & 0.8 & - & 1859.62 & 15.4705dB & 0.227705 \\
    %         max-min & 1000 & 100 & 2 & 1 & 0.8 & - & 1864.2 & 15.4598dB & 0.224875 \\
    %         max-min & 1000 & 100 & 2 & 2 & 0.8 & - & 1859.62 & 15.4705dB & 0.227705 \\
    %         max-min & 1000 & 100 & 2 & 2 & 0.5 & - & 1860.59 & 15.4682dB & 0.227674 \\
    %         max-min & 1000 & 100 & 2 & 2 & 0.9 & - & 1859.62 & 15.4705dB & 0.227705 \\
    %         max-min & 1000 & 100 & 2 & 2 & 0.999 & - & 1979.46 & 15.1993dB & 0.22613 \\
    %         \hline
    %     \end{tabular}
    % \end{center}


    %
}

% BADANIE PERCEPCJI I OCENY OBRAZÓW
% file:///Users/grzegorzkazana/Desktop/Stegano_Ant/Comparisonoffoursubjectivemethodsforimagequalityassessment.pdf
% file:///Users/grzegorzkazana/Desktop/Stegano_Ant/ImageQualityAssessment.pdf
% file:///Users/grzegorzkazana/Desktop/Stegano_Ant/Contrast_masking_in_human_vision.pdf

% SSIM https://www.cns.nyu.edu/pub/eero/wang03-reprint.pdf
% SSIM vs PSNR https://www.semanticscholar.org/paper/Image-Quality-Metrics%3A-PSNR-vs.-SSIM-Hor%C3%A9-Ziou/4a9a98cc86b1e07e548b7edee045275a793f6698

% WYNIKI INNYCH PRAC
% - lsb/dvt
% https://www.researchgate.net/publication/329245499_LSB_Substitution_and_PVD_performance_analysis_for_image_steganography
%
% - jakaś nakoksana metoda
% file:///Users/grzegorzkazana/Desktop/Stegano_Ant/A%20novel%20algorithm%20for%20colour%20image%20steganography%20using%20a%20new%20intelligent%20technique%20based%20on%20three%20phases.pdf
%
% - wariacje IWT + sztosowe porównanie str. 24
%   file:///Users/grzegorzkazana/Desktop/Stegano_Ant/A%20Novel%20Image%20Steganographic%20Method%20based%20on%20Integer%20Wavelet%20Transformation%20and%20Particle%20Swarm%20Optimization.pdf
%
% - ACO steganography str. 6
%   file:///Users/grzegorzkazana/Desktop/Stegano_Ant/Ant%20Colony%20Optimization%20ACO%20based%20Data%20Hiding%20in%20Image%20Complex%20Region.pdf
%
% - vivbs
% file:///Users/grzegorzkazana/Desktop/Stegano_Ant/Varying_index_varying_bits_substitution_algorithm_.pdf

% find docs/thesis/assets/images/experiments/**/*.bmp | ggrep  -oP '^[\w/_:\.,-]+(?=\.bmp)'  | xargs -I XXX convert XXX.bmp XXX.png