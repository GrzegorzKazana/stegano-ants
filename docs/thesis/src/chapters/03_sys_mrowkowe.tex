\chapter{Systemy mrówkowe i mrowiskowe}\label{chap:antsys}
{
    % - krótko czym są algorytmy mrówkowe
    % - czym są systemy wieloagentowe
    % - systemy mrówkowe jako system wieloagentowy
    % - (podobieństwa do algorytmów genetycznych ?)
    % - zastosowanie/rodzaje rozwiązywanych problemów
    % - zasada działania
    %   - problem wyboru krawędzi
    %   - problem aktualizacji śladu feromonowego
    % - wariacje/rodzaje systemów opartych o mrówki
    %   - system mrówkowy (AS - ant system)
    %       - wybór krawędzi
    %       - metody aktualizacji feromonu
    %           - feromon stały
    %           - feromon średni
    %           - feromon cykliczny
    %   - system mrowiskowy (ACS - ant colony system)
    %       - wybór krawędzi
    %       - aktualizacja feromonu
    %   - (system mrowiskowy min-max (MMAS - min max ant system) ?)
    %       - wybór krawędzi
    %       - aktualizacja feromonu
    %   - (system mrowiskowy elitystyczny (EliteAS) ?)
    %       - wybór krawędzi
    %       - aktualizacja feromonu
    %   - (system mrowiskowy rankingowy (RankAS) ?)
    %       - wybór krawędzi
    %       - aktualizacja feromonu
    % - algorytmy mrówkowe w steganografii

    % CO TO
    Systemy mrówkowe (ang. \textit{Ant System - AS}) oraz systemy mrowiskowe (ang. \textit{Ant Colony System - ACS}) są
    metaheurystykami wykorzystywanymi do rozwiązywania trudnych problemów optymalizacyjnych. Ich fundamenty wywodzą się
    z pracy z 1991 roku, zaproponowanej przez M. Dorgio, V. Maniezzo i A. Colorniego \cite{Dorigo1991AntSA}. Jej autorzy
    przedstawili ideę oraz zastosowania algorytmu wzorującego się na zachowaniu mrówek poszukujących pożywania. Od tego
    czasu przedstawiono wiele wariacji oraz ulepszeń algorytmu mrówkowego, a wiele z nich jest obecnie używanych do
    rozwiązywania problemów w bardzo szerokim zakresie dziedzin.

    Ogólna zasada działania systemów mrówkowych opiera się na obserwacji zachowania prawdziwych mrówek eksplorujących
    otoczenie w celu odnalezienie pożywienia. Początkowo, każda mrówka porusza się w chaotyczny i losowy sposób,
    przeszukując najbliższe otoczenie mrowiska. W przypadku, w którym mrówka odnajduje pożywienie, zaczyna ona drogę
    powrotną do mrowiska. Podczas jej przebywania, mrówka niosąca pokarm nanosi na ścieżkę ślad feromonowy. Ma to na
    celu umożliwienie powrotu do obszaru, w którym pokarm został znaleziony. Zarówno mrówka, która odniosła zdobytą
    żywność, jak i pozostałe mrówki w okolicy podczas swojej losowej wędrówki zaczynają faworyzować trasy oznaczone
    śladem feromonowym. Jeśli na jego końcu ponownie zostanie znalezione pożywienie, na drogę powrotną zostanie nałożona
    kolejna warstwa feromonu. Końcowym efektem tego zjawiska jest efekt pozytywnego sprzężenia zwrotnego, gdyż trasy
    obierana przez mrówki jednocześnie stają się dla nich bardziej atrakcyjne.

    Istotnym aspektem w zjawisku odkładania śladu feromonowego jest jego wyparowywanie pod upływem czasu. Jest to
    kluczowa właściwość, przyczyniająca się do zbieżności tras obieranych przez mrówki do optymalnej (najkrótszej) drogi
    prowadzącej do pożywienia. Taki stan rzeczy może być wytłumaczony poprzez analizę ruchu większej liczby mrówek w
    pewnym okresie czasu. Ponieważ przebycie dłuższej ścieżki wymaga więcej czasu, średnio mniej mrówek będzie się nią
    poruszać w stosunku do jednostki odległości. To bezpośrednio przekłada się na mniejszą ilość odłożonego śladu oraz
    jego szybsze odparowanie. Pomysłodawcy systemów mrówkowych podsumowują ich ogólne właściwości wyszczególniając
    następujące cechy \cite{Dorigo1991AntSA}:

    \begin{itemize}
        \item dodatnie sprzężenie zwrotne pozwalające na szybkie odkrywanie dobrych rozwiązań,
        \item rozproszony charakter obliczeń zapobiegający przedwczesnej zbieżności do lokalnego minimum,
        \item zachłanne postępowanie każdej mrówki przyczyniające się do znajdowania akceptowalnych rozwiązań w bardzo
        krótkim czasie.
    \end{itemize}


    % SYSTEMY WIELOAGENTOWE
    \section{Systemy wieloagentowe}
    {
        Ponieważ systemy mrówkowe oraz mrowiskowe można zaliczyć do grupy systemów wieloagentowych (ang.
        \textit{Multi-Agent System - MAS}), istotne jest zrozumienie celów, trudności, zalet oraz ograniczeń tej klasy
        rozwiązań. Fundamentem systemów wieloagentowych jest pojedynczy agent wchodzący w interakcję z starannie
        zaprojektowanym środowiskiem. Pomimo braku ścisłej definicji agenta, która byłaby w stanie objąć wszystkie
        istotne przykłady oraz zastosowania systemów, agent musi spełniać następujące kryteria \cite{Balaji2010AnIT}.

        \begin{itemize}
            \item Zdolność percepcji otoczenia. Sposób postrzegania oraz zakres odbieranych informacji zależy od
            rozwiązywanego problemu i jest proporcjonalny do poziomu złożoności rozwiązywanego problemu.
            \item Autonomia. Każdy agent samoistnie dąży do realizacji swoich celów, nie wymaga interakcji z innymi
            agentami ani zewnętrznej ingerencji człowieka w działanie systemu.
            \item Responsywność i proaktywność. Agent pod wpływem bodźców odbieranych z środowiska podejmuje decyzje
            przybliżające go do realizacji celu.
            \item Komunikacja i zachowanie społeczne. Interakcja pomiędzy osobnikami jest kluczowym elementem systemów
            wieloagentowych. Pomimo że osobniki są w stanie działać autonomicznie, komunikacja pozwala na dzielnie się
            wiedzą i zdobytym doświadczeniem, co przekłada się na szybsze dążenie do lepszych rozwiązań systemu jako
            całości.
            \item Lokalność celu. Agent nie jest w pełni świadomy stanu całego systemu, ani ostatecznego celu jego
            działania. Przeciwnie, agentom przydzielane jest realizacja lokalnych celów, które są dużo prostsze do
            osiągnięcia niż globalny cel działania całego systemu. Poprzez interakcję oraz mnogość agentów osiąganie
            lokalnych celów przekłada się na odkrywanie coraz to lepszych rozwiązań globalnego problemu.
        \end{itemize}

        Ze względu na powyższą charakterystykę oraz ogólną koncepcję systemów wieloagentowych, posiadają one wiele
        zalet, których pozbawione są scentralizowane metody rozwiązywania problemów. Autonomia agentów pozawala na
        zastosowanie metodyk programowania równoległego i lepszego wykorzystania dostępnych zasobów procesora.
        Zapewniają również lepszą skalowalność pod kątem rozmiaru problemu -- dla większych danych wejściowych możliwe
        jest uruchomienie większej liczby agentów zaangażowanych w rozwiązanie zadanego problemu. Lokalność i
        niezależność agentów pozwala na wykorzystanie rozproszonych systemów komputerowych, na przykład klastrów -- to z
        kolei przekłada się na większą niezawodność systemów, gdyż błąd działania pojedynczego agenta nie powoduje
        awarii całego systemu. Rozproszone systemy wieloagentowe mają również swoje wady, należą do nich narzut
        komunikacji agentów, który może utrudniać równoległe wykonywanie operacji, oraz brak gwarancji osiągnięcia
        globalnego celu \cite{Dorri2018MultiAgentSA, Balaji2010AnIT}.

        % PRZYKŁADY/ZASTOSOWANIA SYS WIELOAGENTOWYCH
        W związku z szerokim wachlarzem zalet, systemy wieloagentowe znajdują zastosowanie w rozległym spektrum dziedzin
        i aplikacji. Ich przeznaczenia można zaobserwować poczynając od modelowania problemów zbyt złożonych do
        klasycznej analizy, rozwiązywania zadań wyznaczania drogi, zarówno w logistycznych łańcuchach zaopatrzeń jak i
        w trasowaniu pakietów w sieciach IP, oraz zarządzania i monitorowania systemami, takimi jak sieci energetyczne
        czy platformy chmurowe \cite{Dorri2018MultiAgentSA, Oprea2004ApplicationsOM}.

        % SYS MRÓWKOWE JAKO SYS WIELOAGENTOWE
        Na podstawie omówionych cech systemów wieloagentowych, można wysnuć wiele paralel pomiędzy nimi i systemami
        mrówkowymi. Pojedynczymi agentami są mrówki, które postrzegają środowisko w określony i charakterystyczny sposób
        dla natury problemu. Ich percepcja ogranicza się do postrzegania ich położenia, dostępnych ścieżek oraz śladu
        feromonowego z nimi związanego. Każda mrówka jest w pełni autonomiczna, gdyż nie wymaga dodatkowej interakcji do
        wykonywania działań prowadzących do realizacji lokalnego celu. Mrówki przejawiają zachowania społeczne oraz
        komunikują się za pomocą nanoszonego śladu feromonowego na ścieżki prowadzące do pożywienia. Dzięki tym cechom,
        agenty realizujące lokalne cele, czyli znalezienie pożywienia, przyczyniają się do wspólnego osiągnięcia celu
        globalnego, jakim jest wyznaczenie najkrótszej drogi do niego prowadzącej.
    }

    % ZASTOSOWANIE SYS MRÓWKOWYCH
    \section{Zastosowania systemów mrówkowych}
    {
        Ponieważ metaheurystyka systemu mrówkowego jest oparta na zachowaniu mrówek poszukujących najkrótszej drogi do
        pożywienia, oczywiste zdają się być próby zastosowania jej do problemu komiwojażera, znanego w anglojęzycznej
        literaturze jako \textit{Travelling Salesman Problem - TSP}. Zadaniem postawionym przed algorytmem poszukującym
        rozwiązania \textit{TSP} jest znalezienie najkrótszej drogi łączącej wszystkie $n$ miast w taki sposób, że każde
        miasto zostanie odwiedzone jednokrotnie. Problem komiwojażera jest problemem \textit{NP trudnym}, a
        asymptotyczna złożoność algorytmu przeszukiwania wyczerpującego wynosi $O(n!)$. \textit{TSP} zawdzięcza swoją
        popularność prostocie jego opisu, będącej w opozycji do trudności realizacji jego rozwiązania. Jego pierwsza
        formalna definicja została przedstawiona przez matematyka K. Mengera w roku
        1930 \cite{Mazidi2017MetaHeuristicAF}. Od tego czasu naukowcy z wielu dziedzin starali się zaproponować algorytmy
        i heurystyki pozwalające na znalezienie dobrych rozwiązań w czasie wielomianowym. Pomimo braku sukcesu w
        odkryciu algorytmu pozwalającego na znalezienie optymalnego rozwiązania w czasie wielomianowym, poczyniono
        istotny postęp w tworzeniu algorytmów skupionych na odkrywaniu akceptowalnych rozwiązań w krótszym czasie.

        Jednym z najważniejszych przełomów w badaniach \textit{TSP}, był algorytm zaproponowany przez N. Christofides w
        roku 1976. Odkryte rozwiązanie gwarantuje znalezienie drogi nie dłuższej od optymalnej o $50\%$ w czasie
        $O(n^3)$ \cite{Christofides1976WorstCaseAO}. Od tego czasu, powstawało wiele alternatywnych rozwiązań, lecz żadne
        z nich nie zdołało obniżyć górnej granicy długości drogi w znaczący sposób.

        Alternatywnym podejściem do problemu komiwojażera oraz innych problemów z klasy \textit{NP trudnych}, które
        zyskało na popularności jest stosowanie metaheurystyk, czyli ogólnych schematów i metod przeznaczonych do
        rozwiązywania szerokiej gamy problemów algorytmicznych. Metaheurystyki są najczęściej inspirowane systemami
        występującymi w naturze, i dają dobre rezultaty w problemach optymalizacyjnych problemów o charakterze losowym i
        dynamicznym \cite{Bianchi2008ASO}. Do najszerzej stosowanych należą symulowane wyżarzanie (ang. \textit{Simulated
        Annealing}) \cite{Kirkpatrick1983OptimizationBS}, algorytmy genetyczne (ang. \textit{Genetic
        Algorithms}) \cite{Fraser1957SimulationOG}, metody optymalizacji cząsteczkowej (ang. \textit{Particle Swarm
        Optimization}) \cite{Poli2007ParticleSO} oraz opisywane w tym rozdziale systemy mrówkowe i mrowiskowe. Wszystkie
        z wymienionych metod były wykorzystywane w rozwiązywaniu problemu komiwojażera \cite{Prabakaran2019ASO,
        Mazidi2017MetaHeuristicAF}.

        Pomimo niesprzecznej wagi i istotności \textit{TSP}, systemy mrówkowe i mrowiskowe znalazły zastosowanie w wielu
        innych problemach. Kluczowym etapem decydującym o możliwości i efektywności rozwiązania zadanego problemu przez
        system mrówkowy lub mrowiskowy jest wyznaczenie jego odpowiedniej reprezentacji grafowej \cite{Dorigo1991AntSA}.
        Dodatkową zaletą wynikającą z zastosowania tego rodzaju metaheurystyki jest możliwość rozwiązywania problemów
        dynamicznych, w których warunki ulegają zmianom w trakcie pracy algorytmu. Do problemów rozwiązywanych przez
        systemy mrówkowe można zaliczyć między innymi:

        \begin{itemize}
            \item kwadratowe zagadnienie przydziału (ang. \textit{quadratic assignment problem}) \cite{Maniezzo1999TheAS,
            Gambardella1999AntCF},
            \item harmonogramowanie (ang. \textit{scheduling}) \cite{JSSchedulingColoroni94, Merkle2002AntCO},
            \item marszrutyzacja (ang. \textit{vehicle routing problem}) \cite{Bullnheimer1999ApplyingTA},
            \item trasowanie w sieciach (ang. \textit{routing}) \cite{Caro1999AntNetAM, Bonabeau1998RoutingIT}.
        \end{itemize}
    }

    % ZASADA DZIAŁANIA
    \section{Zasada działania}
    {
        Ponieważ zachowanie mrówek poszukujących najkrótszej ścieżki do pożywienia najłatwiej i najbardziej intuicyjnie
        jest analizować w odniesieniu do problemu komiwojażera, zdecydowano się stosować słownictwo oparte na problemie
        poruszania się po mapie złożonej z miast połączonych drogami o znanej długości. Ważne jest jednak mieć na uwadze
        fakt, że jest to tylko przykład dydaktyczny, a działanie algorytmu można równie trafnie opisać posługując się
        pojęciami grafu, wierzchołków i krawędzi je łączących.

        Pierwszym etapem algorytmu jest umieszczenie mrówek na mapie oraz inicjalizacja struktury reprezentującej ślad
        feromonowy. Mrówki są przydzielane do miast w losowy sposób, a ślad jest inicjowany wartością $\tau_0$. Każda z
        mrówek inicjalizuje strukturę służącą do przechowywania informacji o uprzednio odwiedzonych miastach. Zostaje do
        niej swoje miasto początkowe. Następnie każda mrówka wybiera drogę prowadzącą z bieżącego miasta $i$ do
        kolejnego miasta $j$, pod warunkiem że miasto $j$ nie zostało już odwiedzone. Prawdopodobieństwo każdego
        przejścia w kroku $t$ jest określone funkcją $P_{ij}(t)$. Po wykonaniu pojedynczego kroku przez każdą mrówkę,
        następuje proces nakładania śladu feromonowego. Natężenie nakładanego śladu jest określone za pomocą funkcji
        $\Delta\tau_{ij}(t, t+1)$. Proces wyboru kolejnego miasta i nanoszenia śladu feromonowego jest powtarzany aż do
        momentu w którym mrówki odwiedziły już wszystkie miasta, lub został osiągnięty warunek końcowy. Po zakończeniu
        iteracji, która w przypadku zadań polegających na odwiedzeniu wszystkich miast nazywana jest cyklem, następuje
        ponowne naniesienie śladu feromonowego. Algorytm jest wykonywany do momentu realizacji ustalonej liczby iteracji
        lub braku poprawy rozwiązania \cite{Dorigo1991AntSA}.

        Ogólna postać algorytmu może zostać podsumowana w następujący sposób:

        \begin{enumerate}
            \item \label{step:place} Umieść mrówki na wierzchołkach grafu.
            \item \label{step:dispatch} Każda mrówka dokonuje wyboru kolejnego nieodwiedzonego miasta zgodnie z ustaloną
            funkcją prawdopodobieństwa $P_{ij}(t)$.
            \item \label{step:update} Aktualizuj ślad feromonowy za pomocą funkcji $\Delta\tau_{ij}(t, t+1)$.
            \item Powtórz kroki \ref{step:dispatch}-\ref{step:update} do momentu odwiedzenia wszyskich wierzchołków
            przez każdą mrówkę
            \item \label{step:update_cycle} Aktualizuj ślad feromonowy za pomocą funkcji $\Delta\tau_{ij}(t, t+n)$.
            \item Powtórz kroki \ref{step:place}-\ref{step:update_cycle}.
        \end{enumerate}
    }

    % RODZAJE SYS MRÓWKOWYCH
    \section{Rodzaje systemów mrówkowych}
    {
        Pomysłodawcy systemów mrówkowych, M. Dorgio, V. Maniezzo i A. Colorni, zaproponowali trzy wariacje systemu.
        Każda z nich różni się pod kątem sposobu aktualizacji śladu feromonowego. Są to modele \textit{Ant Density},
        \textit{Ant Quantity} oraz \textit{Ant cycle}. W dalszych częściach pracy stosowane są odpowiednio
        nazwy modelu o feromonie stałym, średnim oraz cyklicznym.

        Cechą wspólną powyższych modeli jest metoda wyboru krawędzi w kroku $t$. Prawdopodobieństwo, że mrówka
        znajdująca się w wierzchołku $i$ wybierze krawędź prowadzącą do wierzchołka $j$ jest określona wzorem
        \ref{eqt:edge_dispatch_propab}.

        \begin{equation}\label{eqt:edge_dispatch_propab}
            P_{ij}(t) = \left\{\begin{matrix}
                0 & j \not\in J\\
                \frac{[\tau_{ij}(t)]^\alpha [\eta_{ij}]^\beta}{\sum_{j\in J} {[\tau_{ij}(t)]^\alpha [\eta_{ij}]^\beta} } & j \in J
                \end{matrix}\right.
        \end{equation}

        gdzie
        \begin{itemize}
            \item $J$ jest zbiorem wierzchołków połączonych krawędzią z wierzchołkiem $i$, które nie zostały jeszcze
            odwiedzone przez mrówkę podejmującą decyzję,
            \item $\tau_{ij}(t)$ jest natężeniem śladu feromonowego krawędzi pomiędzy wierzchołkami $i$ i $j$ w kroku
            $t$,
            \item $\eta_{ij}$ jest współczynnikiem widoczności wierzchołka $j$ z perspektywy wierzchołka $i$. Jego
            wartość jest obliczana jako odwrotność długości krawędzi $\eta_{ij} = \frac{1}{d_{ij}}$,
            \item $\alpha$ i $\beta$ są współczynnikami pozwalającymi kontrolować istotność śladu feromonowego względem
            widoczności.
        \end{itemize}

        Ogólną zasadę śladu feromonowego aktualizacji wyraża wzór \ref{eqt:pheromone_update}. Współczynnik $\rho$ jest
        odpowiedzialny za wyparowywanie śladu, co chroni przez nieskończoną jego akumulacją. Istotne jest, aby jego
        wartość była dodatnia, lecz mniejsza niż jeden $0 < \rho < 1$. Wartość $1 - \rho$ mianuje się współczynnikiem
        wyparowania.

        \begin{equation}\label{eqt:pheromone_update}
            \Delta\tau_{ij}(t + 1) = \rho\tau_{ij}(t) + \Delta\tau_{ij}(t, t + 1)
        \end{equation}

        To co rozróżnia opisywane metody, jest reguła wyznaczania przyrostu śladu feromonowego $\Delta\tau_{ij}(t, t +
        1)$.

        % FEROMON STAŁY
        \subsection{Model feromon stały}
        {
            W przypadku modelu o feromonie stałym, przyrost jest stałą wartością $Q$ dla krawędzi którą mrówka zdecyduje
            się przebyć w drodze pomiędzy wierzchołkami $i$ i $j$. Po zakończeniu kroku lub cyklu, wartości te są
            sumowane dla każdej z $m$ mrówek -- jeśli krawędź została przemierzona przez więcej niż jedną mrówkę,
            przyrost śladu feromonowego będzie proporcjonalnie większy i jest wyrażony wzorem
            \ref{eqt:pheromone-density}.

            \begin{equation}\label{eqt:pheromone-density}
                \Delta\tau_{ij}(t, t + 1) = \sum_{k=1}^m \left\{
                        \begin{matrix}
                            Q & (i, j) \in V_k\\
                            0 & w\ przeciwnym\ wypadku
                        \end{matrix}
                    \right.
            \end{equation}
            gdzie $V_k$ jest krawędzią wybrana przez $k$ mrówkę w kroku $t$.
        }

        % FEROMON ŚREDNI
        \subsection{Model feromon średni}
        {
            W modelu o feromonie średnim (\textit{Ant Quantity}), przyrost śladu feromonowego jest odwrotnie
            proporcjonalny do długości krawędzi łączącej wierzchołki\ref{eqt:pheromone-quantity}. Powoduje to dodatkowe
            faworyzowanie widoczności przy wyborze krawędzi, gdyż krótsze krawędzie będą stawać się bardziej atrakcyjne
            dla innych mrówek.

            \begin{equation}\label{eqt:pheromone-quantity}
                \Delta\tau_{ij}(t, t + 1) = \sum_{k=1}^m \left\{
                        \begin{matrix}
                            \frac{Q}{d_{ij}} & (i, j) \in V_k\\
                            0 & w\ przeciwnym\ wypadku
                        \end{matrix}
                    \right.
            \end{equation}
        }

        % FEROMON CYKLICZNY
        \subsection{Model feromon cykliczny}
        {
            Ostatnim zaproponowanym wariantem systemu mrówkowego jest model feromonu cyklicznego (\textit{Ant Cycle}).
            Różni się on znacząco od poprzednich, gdyż feromon jest aktualizowany jednokrotnie w każdej iteracji
            algorytmu, a nie po każdym kroku. Wyznaczenie nowego śladu następuje po $n$ krokach, gdzie $n$ jest
            długością cykli pokonywanych przez mrówki. Przyrost śladu feromonowego $\Delta\tau_{ij}(t, t + n)$ jest
            obliczany dla każdej krawędzi trasy pokonanej przez każdą z $m$ mrówek, i jest odwrotnie proporcjonalny do
            długości całej trasy. Uzasadnieniem takiego postępowania jest intuicja mówiąca, że znajomość globalnej oceny
            trasy (jej długości) pozwoli lepiej ocenić każde z należących do niej kroków. Regułę obliczania przyrostu
            śladu wyraża wzór \ref{eqt:pheromone-cycle}. $L_k$ oznacza zbiór krawędzi należących do trasy pokonanej
            przez mrówkę $k$.

            \begin{equation}\label{eqt:pheromone-cycle}
                \Delta\tau_{ij}(t, t + n) = \sum_{k=1}^m \left\{
                        \begin{matrix}
                            \frac{Q}{||L^k||} & (i, j) \in L_k\\
                            0 & w\ przeciwnym\ wypadku
                        \end{matrix}
                    \right.
            \end{equation}

            Twórcy powyższych algorytmów z powodzeniem przeprowadzili eksperymenty, których celem było zbadanie ich
            efektywności w rozwiązywaniu problemu komiwojażera. W przypadku każdego z modeli osiągnięto zadowalające
            wyniki, lecz najszybszą zbieżność do optymalnych rozwiązań zaobserwowano w modelu z feromonem cyklicznym.
            Podczas eksperymentów, autorzy uzyskali ówcześnie znane najlepsze rozwiązanie \textit{TSP} dla zbioru
            Oliver30, lecz nie odnieśli sukcesu w problemach większego rozmiaru -- Eilon50 i Elion75. Jednakże, w
            artykule wyraźnie podkreślono, że osiągnięcie najlepszych wyników w problemie komiwojażera nie było głównym
            celem pracy, a problem ten został wybrany jedynie w celach dydaktycznych i porównawczych. Zwrócono
            jednocześnie uwagę na istotność procesu doboru parametrów algorytmu, takich jak liczba mrówek i współczynnik
            wyparowania śladu feromonowego. Zaobserwowano, że dla większości rozwiązywanych problemów osiągano najlepsze
            rezultaty gdy liczba mrówek $m$ jest zbliżona do liczby wierzchołków $n$ grafu reprezentującego problem. Dla
            wariantu algorytmu z feromonem cyklicznym, najlepsze rezultaty osiągano dla współczynnika odparowania $(1 -
            \rho)$ bliskiemu $0.5$, i wartości $\alpha$ i $\beta$ będących w stosunku $\frac{\beta}{\alpha}$ zbliżonym
            do zakresu $[2.5, 5]$  \cite{Dorigo1991AntSA}.
        }

        % SYS MROWISKOWY
        \subsection{System mrowiskowy}
        {
            Znaczącym krokiem naprzód w dziedzinie algorytmów mrówkowych, jest system mrowiskowy zaproponowany przez M.
            Dorigo, jednego z twórców systemów mrówkowych, oraz L. Gambardella w roku 1997 \cite{Dorigo1997AntCS}. Tym
            razem głównym celem pracy było opracowanie zmodyfikowanej wersji algorytmu która mogłaby konkurować z
            najlepszymi znanymi algorytmami służącymi do rozwiązywania problemu komiwojażera o dużym rozmiarze
            wejściowym.

            Bazując na doświadczeniu i wynikach eksperymentów przeprowadzonych za pomocą klasycznych systemów
            mrówkowych, zaproponowano wprowadzenie trzech istotnych zmian w zakresie działania algorytmu.

            \begin{itemize}
                \item Rozszerzenie reguły wyboru krawędzi łączącej miasta $i$ i $j$ o dwa tryby działania -- eksplorację
                i eksploatację. Wybór pomiędzy trybem działania jest czyniony na podstawie wartości zmiennej losowej $q$
                przyjmującej wartości z zakresu $[0, 1]$. Jeśli jej wartość jest mniejsza bądź równa wartości parametru
                algorytmu $q_0$, mrówka wybierze krawędź maksymalizującą wartość iloczynu widoczności i natężenia śladu
                feromonowego -- jest to strategia eksploatacji. W przeciwnym razie, kolejne miasto $s$ zostanie wybrana
                zgodnie z funkcją prawdopodobieństwa zbliżoną do zaproponowanej w systemie mrówkowym. Wzór opisujący
                wybór kolejnego miasta określa wzór \ref{eqt:ant_colony_dispatch}.

                \item Wprowadzanie globalnej reguły aktualizacji śladu feromonowego, która jest aplikowana po
                zakończeniu każdego cyklu algorytmu. Jest to rozszerzenie wariantu systemu mrówkowego typu \textit{Ant
                cycle}, z tą różnicą że aktualizacji podlegają jedynie krawędzie należące do najlepszej znalezionej
                trasy. Wpływ na istotność przyrostu śladu jest zależny od osobnego współczynnika odparowania $\alpha$
                oraz długości najdłuższej trasy $L_{gb}$. Zależność tą wyraża wzór \ref{eqt:ant_colony_global_update_1}
                i \ref{eqt:ant_colony_global_update_2}.

                \item Usprawnienie lokalnej reguły aktualizacji śladu feromonowego. Według eksperymentów, globalna
                regułą nie nie jest wystarczająca w zapewnianiu akceptowalnych rezultatów. Autorzy testowali różne
                metody wyznaczania wartości przyrostu śladu feromonowego, w tym metody czerpiące inspirację z
                wzmacnianych metod uczenia maszynowego \textit{Q-learning} \cite{Watkins1989LearningFD}. Równie dobre
                rezultaty osiągnięto stosując prostszą obliczeniowo regułę opisaną wzorem
                \ref{eqt:ant_colony_local_update_1} oraz \ref{eqt:ant_colony_local_update_2}. Wartość $\rho$ jest
                współczynnikiem wyparowania niezależnym od $\alpha$, a $V_k$ oznacza krawędź wybraną przez mrówkę $k$.
            \end{itemize}

            \begin{equation}\label{eqt:ant_colony_dispatch}
                s = \left\{
                        \begin{matrix}
                            argmax_{u \in J} \{\tau_{ij}\eta_{ij}^\beta\} & q \leq q_0\\
                            \frac{\tau_{ij} \eta_{ij}^\beta}{\sum_{j\in J} {\tau_{ij} \eta_{ij}^\beta} } & w\ przeciwnym\ wypadku
                        \end{matrix}
                    \right.
            \end{equation}

            \begin{equation}\label{eqt:ant_colony_global_update_1}
                \tau_{ij}(t + n) = (1 - \alpha) \cdot \tau_{ij}(t) + \alpha \cdot \Delta\tau_{ij}(t, t + n)
            \end{equation}

            \begin{equation}\label{eqt:ant_colony_global_update_2}
                \Delta\tau_{ij}(t, t + n) = \left\{
                    \begin{matrix}
                        \frac{1}{L_{gb}} & (i, j) \in L_{gb} \\
                        0 & w\ przeciwnym\ wypadku
                    \end{matrix}
                \right.
            \end{equation}

            \begin{equation}\label{eqt:ant_colony_local_update_1}
                \tau_{ij}(t + 1) = (1 - \rho) \cdot \tau_{ij}(t) + \rho \cdot \Delta\tau_{ij}(t, t + 1)
            \end{equation}

            \begin{equation}\label{eqt:ant_colony_local_update_2}
                \Delta\tau_{ij}(t, t + 1) = \sum_{k=1}^m \left\{
                    \begin{matrix}
                        \tau_0 & (i, j) \in V_k \\
                        0 & w\ przeciwnym\ wypadku
                    \end{matrix}
                \right.
            \end{equation}

            Autorom udało się dowieść, że zaproponowany system mrowiskowy jest w stanie osiągać równie dobre, lub w
            znacznej części badanych problemów lepsze wyniki niż inne heurystyki. W problemach o rozmiarze 75 i 100
            miast, uzyskano lepsze rezultaty niż za pomocą algorytmów genetycznych, regularyzacji metodą elastycznej
            sieci oraz symulowanego wyżarzania. Dodatkowo, autorzy zgłębili temat i osiągalne rezultaty wykorzystania
            heurystyki systemu mrowiskowego jako generatora tras wejściowych do algorytmów lokalnej wyczerpującej
            optymalizacji, takich jak \textit{3-opt}.
        }

        % SYS MIN-MAX
        \subsection{System mrówkowy Max-Min}
        {
            Alternatywnym ulepszeniem klasycznego algorytmu mrówkowego, jest system mrówkowy max-min (\textit{Max-Min
            Ant System - MMAS}) \cite{Sttzle2000MAXMINAS}. Podobnie jak w przypadku pracy \textit{Ant colony system: a
            cooperative learning approach to the traveling salesman problem} \cite{Dorigo1997AntCS}, głównym celem
            autorów było zaproponowanie wariantu algorytmu mrówkowego który gwarantuje lepsze wyniki dla grafów
            większych rozmiarów.

            Autorzy zaproponowali następujące usprawnienia względem systemu mrówkowego opisanego przez M. Dorgio w 1991
            roku.

            \begin{itemize}
                \item Po każdej iteracji algorytmu, jedynie mrówka która przebyła najkrótszą trasę nanosi ślad
                feromonowy. Istnieją dwie wariacje tej zasady, według pierwszej wybierana jest najkrótsza trasa w
                bieżącej iteracji (ang. \textit{iteration best}), a w drugiej wybierana jest najlepsza trasa znaleziona
                w całkowitym czasie działanie algorytmu (ang. \textit{global best}). Wzór opisujący regułę aktualizacji
                śladu feromonowego jest analogiczny do wzoru wyznaczającego przyrost podczas globalnej aktualizacji w
                systemie mrowiskowym \ref{eqt:ant_colony_global_update_2}.

                \item W celu uniknięcia stagnacji algorytmu, polegającej na wybieraniu przez wszystkie mrówki tej samej
                trasy, wartości śladu feromonowego są ograniczone do wartości będących parametrami działania algorytmu
                $[\tau_{min}, \tau_{max}]$.

                \item Ślad feromonowy jest inicjalizowany wartością $\tau_{max}$. Uzasadnieniem takiego wyboru jest
                skłonienie mrówek do śmielszej eksploracji nieznanych rozwiązań na początku działania algorytmu. Takie
                działanie początkowo zmniejsza znaczenie widoczności miast podczas wyboru krawędzi.
            \end{itemize}
        }
    }

    % PODSUMOWANIE
    \section{Podsumowanie}
    {
        Pomimo obszerności powyższego porównania istniejących systemów mrówkowych i mrowiskowych, nie jest ono w żadnym
        stopniu wyczerpujące. Powstało wiele innych rozwiązań charakteryzujących się obiecującymi rezultatami. Należą do
        nich przykładowo systemy elitystyczne (ang. \textit{Elitist Ant System - EAS}) \cite{Dorigo1996AntSO} oraz
        systemy rankingowe (ang. \textit{Rank Ant System - ASRank}) \cite{Bullnheimer1997ANR}. Jak można ocenić,
        dziedzina systemów mrówkowych jest wciąż aktualna w rozwiązywaniu problemów, których rozwiązania są nieosiągalne
        przy stosowaniu algorytmów wyczerpujących i zachłannych \cite{Dorigo2003TheAC}.
    }
}

% CO TO
% file:///Users/grzegorzkazana/Desktop/Stegano_Ant/alg_mrow.pdf
% file:///Users/grzegorzkazana/Desktop/Stegano_Ant/Bura_Wielokryterialne_mrowiskowe_algorytmyprom.Boryczka.pdf
% file:///Users/grzegorzkazana/Desktop/Stegano_Ant/Ro%CC%81zne%20reprezentacje%20mapy%20feromonowej...jd_aco_mkp_promotor-Boryczko.pdf
% file:///Users/grzegorzkazana/Desktop/Stegano_Ant/AlgorytmyIProgramownieMrowiskowe.pdf
% file:///Users/grzegorzkazana/Desktop/Stegano_Ant/MODEL%20TEORETYCZNY%20ALGORYTMU%20MRO%CC%81WKOWEGO%20SAS.pdf
% https://www.diva-portal.org/smash/get/diva2:1214402/FULLTEXT01.pdf
% file:///Users/grzegorzkazana/Desktop/Stegano_Ant/TheAntColonyOptimizationMetaheuristic_AlgorithmsApplicationsandAdvances.pdf

% SYSTEMY WIELOAGENTOWE
% http://www.masfoundations.org/mas.pdf
% https://www.researchgate.net/publication/226165258_An_Introduction_to_Multi-Agent_Systems - Balaji2010AnIT
% https://link.springer.com/content/pdf/10.1007%2F1-4020-8159-6_9.pdf - Oprea2004ApplicationsOM

% PRZYKŁADY/ZASTOSOWANIA SYS WIELOAGENTOWYCH
% file:///Users/grzegorzkazana/Desktop/Stegano_Ant/TheAntColonyOptimizationMetaheuristic_AlgorithmsApplicationsandAdvances.pdf
% https://www.researchgate.net/publication/235439153_Ant_sytem_Optimization_by_a_colony_of_cooperating_agents
% https://www.researchgate.net/publication/226165258_An_Introduction_to_Multi-Agent_Systems
% https://en.wikipedia.org/wiki/Multi-agent_system
% https://link.springer.com/content/pdf/10.1007%2F1-4020-8159-6_9.pdf

% ZASTOSOWANIE SYS MRÓWKOWYCH
% file:///Users/grzegorzkazana/Desktop/Stegano_Ant/TheAntColonyOptimizationMetaheuristic_AlgorithmsApplicationsandAdvances.pdf-Dorigo2003TheAC

% ZASADA DZIAŁANIA
% file:///Users/grzegorzkazana/Desktop/Stegano_Ant/alg_mrow.pdf
% file:///Users/grzegorzkazana/Desktop/Stegano_Ant/Ro%CC%81zne%20reprezentacje%20mapy%20feromonowej...jd_aco_mkp_promotor-Boryczko.pdf
% file:///Users/grzegorzkazana/Desktop/Stegano_Ant/MODEL%20TEORETYCZNY%20ALGORYTMU%20MRO%CC%81WKOWEGO%20SAS.pdf
% file:///Users/grzegorzkazana/Desktop/Stegano_Ant/AlgorytmyIProgramownieMrowiskowe.pdf

% RODZAJE SYS MRÓWKOWYCH
% file:///Users/grzegorzkazana/Desktop/Stegano_Ant/MODEL%20TEORETYCZNY%20ALGORYTMU%20MRO%CC%81WKOWEGO%20SAS.pdf
% file:///Users/grzegorzkazana/Desktop/Stegano_Ant/AlgorytmyIProgramownieMrowiskowe.pdf
% ant cycle i tym podobne
% https://www.researchgate.net/publication/239566982_Ant_System_An_Autocatalytic_Optimizing_Process_Technical_Report_91-016

% SYS MROWISKOWY
% https://people.idsia.ch//~luca/acs-ec97.pdf - Dorigo1997AntCS

% SYS MIN-MAX
% https://www.cs.ubc.ca/~hoos/Publ/fgcs00.pdf - Sttzle2000MAXMINAS

% ALG MRÓWKOWE W STEGANOGRAFII
% file:///Users/grzegorzkazana/Desktop/Stegano_Ant/Ant%20colony%20optimization%20with%20horizontal%20and%20vertical%20crossover%20search%20Fundamental%20visions%20for%20multi-thres.image%20segmentation.pdf
% file:///Users/grzegorzkazana/Desktop/Stegano_Ant/A%20Novel%20Technique%20for%20Steganography%20Method%20Based%20on.pdf
% file:///Users/grzegorzkazana/Desktop/Stegano_Ant/Ant%20Colony%20Optimization%20ACO%20based%20Data%20Hiding%20in%20Image%20Complex%20Region.pdf
% file:///Users/grzegorzkazana/Desktop/Stegano_Ant/Ant%20Colony%20Optimization%20To%20Enhance%20Image%20Steganography.pdf
% file:///Users/grzegorzkazana/Desktop/Stegano_Ant/HIGH%20CAPACITY%20AND%20OPTIMIZED%20IMAGE%20STEGANOGRAPHY%20TECHNIQUE%20BASED%20ON%20ANT%20COLONY%20OPTIMIZATION%20ALGORITHM.pdf
% file:///Users/grzegorzkazana/Desktop/Stegano_Ant/New%20steganography%20algorithm%20to%20conceal%20a%20large%20amount%20of%20secret%20message%20using%20hybrid%20adaptive%20neural.pdf
