
\chapter{Steganografia}\label{chap:steganography}
{
    % - czym jest steganografia
    % - odniesienie do kryptografii
    % - rys historyczny
    %   - pierwsze użycia w starożytności
    %   - przykłady
    % - steganografia cyfrowa (jakiś diagram idk), opis mediów nośnych
    % - steganografia z wykorzystaniem obrazów
    %   - podział na techniki dziedzinowe i przestrzenne
    %   - opis metod (LSB etc)
    %   - vLSB
    %   - wpływ doboru obszaru na pojemność obrazu

    Steganografia jest dziedziną nauki poświęconą ukrywaniu informacji w jawnych kanałach komunikacji.
    Nazwa nauki wywodzi się z języka greckiego, i może być tłumaczona jako 'ukryte pismo' (\textit{steganós} - ukryty, \textit{graphia} - pismo).

    W celu podkreślenia cech oraz charakteru metod steganograficznych często przytaczane jest również pojęcie kryptografii.
    Obiektem zainteresowań kryptografii jest uniemożliwienie zrozumienia treści wiadomości przez osoby postronne,
    pozbawione do niej dostępu. Współcześnie jest to osiągane poprzez stosowanie klucza dzielonego przez osoby zaufane
    (kryptografia symetryczna) lub par kluczy publicznych i prywatnych (kryptografia asymetryczna). Dzięki ich zastosowaniu,
    osoba postronna pomimo dostępu do szyfrogramu nie jest w stanie wydobyć tekstu jawnego.

    Celem technik steganograficznych jest umożliwienie uczestnikom komunikacji przesyłania informacji bez ujawniania faktu istnienia samego przekazu.
    
    % https://www.petitcolas.net/fabien/publications/ieee99-infohiding.pdf
    Pierwszych przykładów zastosowania steganografii można doszukiwać się już w czasach starożytnych.
    Herodotus, w swoim dziele \textit{Dzieje} opisuje historię greckiego polityka Histiajosa,
    który w celu przekazania poufnej informacji wytatuował ją na skalpie zaufanego niewolnika.
    Gdy jego włosy odrosły, został on wysłany w celu doręczenia listu oraz ukrytej wiadomości.
    Do innych przykładów steganografii można zaliczyć również ukrywanie wiadomości w zapisach nutowych,
    stosowanie atramentów sympatycznych lub technikę mikrokropek,
    która przeżyła swój renesans w czasach zimnej i drugiej wojny światowej.
    
    % https://www.researchgate.net/publication/296695903_Watermarking_Security_-_Fundamentals_Secure_Designs_and_Attacks
    % https://dl.acm.org/doi/10.1145/3206004.3206019
    Warto również podkreślić, że kolejnym z zastosowań steganografii są znaki wodne oraz symbole
    pozwalające na identyfikację źródła informacji. Za równo w przypadku multimediów objętych prawami autorskimi
    jak i poufnych dokumentów, ich producent lub organizacja strzegąca ich tajności może umieścić ukryte
    informacje pozwalające wskazać źródło wycieku.
    Podobną technikę stosują producenci drukarek - seria oraz model drukarki może być odzwierciedlona w drukowanym
    dokumencie poprzez układ niewidocznych gołym okiem żółtych kropek. Takie działania mają na celu ułatwienia
    walki z przestępczością polegającą na fałszowaniu dokumentów i banknotów.

    %  STEGANOGRAFIA CYFROWA
    Wraz z wzrostem wykorzystania komputerów do celach multimedialnych oraz rozpowszechnieniu szerokopasmowego
    internetu coraz popularniejsza i bardziej opłacalna staje się steganografia cyfrowa.
    Jej ideą jest wykorzystanie jako medium nośnego różnych rodzajów plików komputerowych
    lub protokołów komunikacji cyfrowej.
    % https://www.researchgate.net/publication/337304290_Covert_Channel_and_Data_Hiding_in_TCPIP
    Przykładem wykorzystania protokołów do celów steganograficznych może być ukrywanie danych w polach kontrolnych ramek TCP/IP,
    kontrolowanie opóźnień między poszczególnymi pakietami lub nawet umyślne powodowanie utrat wybranych pakietów.

    Znacznie prostszą, lecz bardzo rozwiniętą techniką jest wykorzystanie plików multimedialnych takich jak zdjęcia,
    pliku muzyczne i filmy. Do ich szczególnej atrakcyjności jako medium służące do ukrywania informacji przyczynia
    się między innymi ich wszechobecność, duże rozmiary oraz wysoka nadmiarowość.
    Ostatni aspekt w kontekście steganografii ma szczególne znaczenie, gdyż oznacza że modyfikacja pewnej części informacji
    bitowej zawartej w pliku ma niski wpływ na jego końcową treść. Przykładowo, zmiana wartości jednego z kanałów konkretnego
    piksela będzie miało mało zauważalny wpływ na końcowy obraz nawet przez uważnego obserwatora.
    % https://scholarworks.rit.edu/cgi/viewcontent.cgi?article=1305&context=other
    Podobne prawidłowości można również dostrzec w plikach muzycznych - manipulacja zawartością częstotliwości składowych
    będących poza granicą percepcji, czyli poniżej 20Hz i powyżej 20kHz również będzie trudna w detekcji przez subiektywnego odbiorcę.
    Innym przykładem ukrywania informacji w muzyce jest tzw. \textit{backmasking}, polegający na ukrywaniu wiadomości
    możliwych w odbiorze tylko i wyłącznie po odtworzeniu utworu od tyłu. Jednym z pierwszych zespołów przyczyniających się
    do wzrostu popularności powyższych eksperymentów jest \textit{The Beatles}.
 
    %   TECHNIKI UKRYWANIA INFORMACJI W OBRAZACH
    Mimo tego, że jako medium steganograficzne można wykorzystać każdy plik binarny,
    szczególnie dużo uwagi zostało poświęcone cyfrowym obrazom i zdjęciom.
    Pomimo pozornej prostoty powyższego zadania powstało wiele wyrafinowanych metod i technik, różniących się za równo pod kątem
    założeń jak i rezultatów. 
    Pierwszym parametrem mogącym służyć do podziału zaproponowanych metod jest dziedzina w której 
    obraz zostaje poddany analizie. 

    %   TECHNIKI PRZESTRZENNE
    W technikach przestrzennych, obraz jest traktowany jak zbiór punktów (pikseli) umieszczonych w dwuwymiarowym układzie współrzędnych.
    Zaletą tych metod jest ich intuicyjność oraz przystępność, lecz są to również metody bardziej podatne na ataki polegające na wykryciu
    lub zniszczeniu ukrytej wiadomości. % file:///Users/grzegorzkazana/Desktop/Stegano_Ant/AnInclusiveStudyandAnalysis.pdf

    Najbardziej powszechną techniką przestrzenną jest metoda \textit{Least Significant Bit (LSB)}. Jej zastosowanie sprowadza się
    do zastąpienia najmniej znaczącego bitu obrazu będącego nośnikiem informacji bitem wiadomości ukrywanej. W zależności od spadku
    jakości który uznajemy za akceptowalny, możemy wykorzystać \textit{n} najmniej znaczących bitów każdego z kanałów \textit{RGB}.

    Pewnym uszczegółowieniem \textit{LSB} jest metoda \textit{4LSB}. Zakłada ona wykorzystanie dokładnie 4 bitów z każdego bajtu
    obrazu maskującego, co przekłada się na wykorzystanie 50\% pojemności nośnika. Kosztem tak znacznej pojemności jest znaczący spadek
    jakości obrazu oraz ułatwiona steganoanaliza.

    W celu zmniejszenia wykrywalności manipulacji obrazu przy jednoczesnym zachowaniu względnie dużej pojemności steganogramu,
    zaproponowano technikę \textit{Variable Least Significant Bit (VLSB)}. W przeciwieństwie do \textit{LSB} podczas ukrywania danych
    wykorzystywana jest różna ilość bitów obrazu w zależności od położenia piksela.
    Nośnik zostaje podzielony na zadaną ilość sekcji, a następnie dla każdej z nich wyznaczana jest ilość bitów które zostaną
    zastąpione tekstem jawnym. Twórcy metody zaproponowali algorytm \textit{Decreasing Distance Decreasing Bits Algorithm (DDDBA)},
    który na podstawie odległości sektora względem piksela referencyjnego, którym najczęściej jest środkowy piksel obrazu, wyznacza proporcjonalną
    ilość ukrywanych bitów. Wynikiem działania algorytmu \textit{VLSB} jest obraz, którego środkowa część jest mniej zniekształcona,
    co obniża subiektywne odczucie spadku jakości i pozwala na ukrycie większej ilości danych. % https://www.researchgate.net/publication/250002424_Implementation_of_Variable_Least_Significant_Bits_Steganography_Using_DDDB_Algorithm

    % file:///Users/grzegorzkazana/Desktop/Stegano_Ant/Varying_index_varying_bits_substitution_algorithm_.pdf
    Dalszym udoskonaleniem, za równo pod kątem bezpieczeństwa ukrywanej informacji jak i utrudnienia wykrywalności ukrytego przekazu
    jest metoda o nazwie \textit{Varying Index Varying Bits Substitution (VIVBS)} zaproponowana przez XYZ. % TODO: uzupełnić autora
    Podobnie jak w metodzie \textit{VLSB}, ilość wykorzystanych bitów obrazu nośnego jest zmienna. W przeciwieństwie do wariantu \textit{VLSB}
    opartego o algorytm \textit{DDDBA}, ilość bitów ukrywanych w danym pikselu nie jest wyznaczana podczas działania algorytmu, lecz jest
    zależna od dodatkowego klucza będącego parametrem jego działania. Klucz przyjmuje postać tablicy przypisującej indeksowi każdego z
    pikseli ilość bitów które należy zastąpić bitami tekstu jawnego. Ponieważ ilość możliwych kombinacji rozmieszczenia bitów informacji
    w obrazie jest znacząca, odczytanie przekazu poprzez wykorzystanie przeszukiwania wyczerpującego poprzez osobę postronną nieposiadającą
    klucza będzie praktycznie niemożliwe. Główną wadą, powyższej metody jest jej również największa zaleta - klucz definiujący
    rozmieszczenie informacji w pikselach obrazu. Każdorazowe kodowanie informacji wymaga utworzenia klucza, od którego będzie również
    zależeć wpływ procesu na jakość obrazu - arbitralny wybór wysokiej ilości wykorzystanych bitów w nieodpowiednich sekcjach obrazu
    może przykuć uwagę osób postronnych i zdradzić fakt istnienia ukrytego przekazu. Dodatkową wadę metody jest również rozmiar klucza - 
    w minimalnym przypadku, w którym wykorzystujemy 0 lub 1 bit każdego piksela klucz ma rozmiar $w * h$ bitów, gdzie
    \textit{w} i \textit{h} to odpowiednio szerokość i wysokość obrazu. Wraz z wzrostem ilości wykorzystywanych bitów, rozmiar klucza
    również będzie się powiększał.

    % file:///Users/grzegorzkazana/Desktop/Stegano_Ant/ASteganographicMethodForImagesByPixelValueDifferencing.pdf
    W celu poprawy subiektywnej oceny jakości obrazów oraz zmniejszenia ryzyka przekazu w roku XYVZ, XYZ zaproponował metodę % TODO: uzupełnić autora i datę
    \textit{Value Pixel Differencing (VPD)}. Jednym z jej założeń jest uzależnienie ilości wykorzystanych bitów obrazu nośnego
    od różnicy pomiędzy poziomami intensywności kolejnych pikseli. W metodzie \textit{VPD} piksele są odczytywane parami sekwencyjnie,
    zakreślając ciągły, łamany kształt. Dla każdej napotkanej pary pikseli obliczana jest ich różnica jasności, a następnie na jej
    podstawie wyznaczana jest ilość bitów które zostaną podmienione na treść ukrywanej wartości. Ilość ukrytych bitów jest proporcjonalna
    do zmiany wartości pikseli. W ten sposób możliwe jest osiągnięcie obrazów mniej podatnych na steganoanalizę przy jednoczesnym
    zachowaniu znacznej pojemności.

    %   TECHNIKI CZĘSTOTLIWOŚCIOWE
    % file:///Users/grzegorzkazana/Desktop/Stegano_Ant/IntegerWaveletTransform.pdf
    % file:///Users/grzegorzkazana/Desktop/Stegano_Ant/DiscreteCosineTransform.pdf
    % file:///Users/grzegorzkazana/Desktop/Stegano_Ant/A%20Novel%20Image%20Steganographic%20Method%20based%20on%20Integer%20Wavelet%20Transformation%20and%20Particle%20Swarm%20Optimization.pdf
    % file:///Users/grzegorzkazana/Desktop/Stegano_Ant/A%20steganographic%20method%20based%20upon%20JPEG%20and%20particle%20swarm%20optimization%20algorithm.pdf
    Alternatywnym podejściem do steganografii wykorzystującej cyfrowe obrazy, są techniki oparte o częstotliwościową reprezentację obrazów.
    Metody te polegają na transformacji obrazu w postaci bitmapy do macierzy współczynników określających amplitudę lub natężenie fal
    o konkretnych częstotliwościach występujących w obrazie.
    % https://www.sciencedirect.com/science/article/pii/S1665642314716128
    Metody częstotliwościowe zyskały również na popularności w pokrewnej dziedzinie do steganografii, jaką jest oznaczanie
    (ang. \textit{watermarking}) produktów cyfrowych. Za równo ukrywając dane poufne, jak i cyfrowe podpisy chroniące praw autorskich,
    celem metod działających w dziedzinie częstotliwościowej jest nie tylko zachowanie możliwie najwyższej jakości medium w którym
    zawarto dodatkowe informacje, lecz również uodpornienie ukrytego przekazu na jego manipulację i zniszczenie przez kompresję.

    Jedną z transformat pozwalających na wyznaczenie częstotliwościowej reprezentacji obrazu jest \textit{Dyskretna transformata kosinusowa (DCT)}.
    Polega ona na podziale obrazu na bloki, najczęściej rozmiaru 8x8 pikseli. Następnie wyznaczana jest macierz współczynników za pomocą wzoru XYZ. % wzór 4 file:///Users/grzegorzkazana/Desktop/Stegano_Ant/DiscreteCosineTransform.pdf
    W kontekście złożoności obliczeniowej istotny jest fakt, że macierz transformacji można wyznaczyć jednokrotnie i ponownie wykorzystać dla każdego
    z bloków obrazu. W tym przypadku stosowane jest równanie XYZ. % wzór 5 file:///Users/grzegorzkazana/Desktop/Stegano_Ant/DiscreteCosineTransform.pdf
    Jednym z zastosowań dyskretnej transformacji kosinusowej jest stratna kompresja danych, przykładowo w formacie JPEG.
    Po wyznaczeniu współczynników odpowiadających kolejnym częstotliwościom następuje proces kwantyzacji - macierz współczynników zostaje skalarnie przemnożona
    przez macierz kwantyzacji a następnie zaokrąglana. Zadaniem macierzy kwantyzacji jest przeskalowanie współczynników odpowiadających zakresom częstotliwości
    w taki sposób, aby pasma których zmiany mają najmniejszy wpływ na subiektywną percepcję przyjęły wartości bliskie zeru. Ostatnim krokiem kompresji jest
    zaokrąglenie uzyskanych współczynników. Poprzez odrzucenie miejsc po przecinku wszystkich współczynników końcowy obraz charakteryzuje się dużo
    mniejszym rozmiarem. W celu rekonstrukcji wykonuje się transformację odwrotną (\textit{IDCT}).

    % https://www.researchgate.net/publication/228850853_Data_hiding_in_JPEG_images
    % file:///Users/grzegorzkazana/Desktop/Stegano_Ant/A%20steganographic%20method%20based%20upon%20JPEG%20and%20particle%20swarm%20optimization%20algorithm.pdf
    Przykładem algorytmu steganograficznego korzystającego z \texti{DCT} jest praca XYZ autorstwa XYZ. Wyznaczone współczynniki częstotliwości służą za nośnik
    tekstu jawnego - współczynnikom z eksperymentalnie dobranych pasm częstotliwości zostają podmienione najmniej znaczące bity, podobnie jak w przestrzennych
    metodach \textit{LSB}. W celu osiągnięcia dużo większej odporności steganogramu na kompresję, XYZ zaproponował metodę opartą o manipulację znaków współczynników,
    w odróżnieniu do manipulacji ich wartościami. % https://ieeexplore.ieee.org/abstract/document/8901147

    % file:///Users/grzegorzkazana/Desktop/Stegano_Ant/ICME05-Lossless20Data20Hiding20Using20Integer20Wavelet20Transform20and20Threshold20Embedding20Technique.pdf
    % file:///Users/grzegorzkazana/Desktop/Stegano_Ant/A%20Novel%20Image%20Steganographic%20Method%20based%20on%20Integer%20Wavelet%20Transformation%20and%20Particle%20Swarm%20Optimization.pdf
    Inną transformacją wykorzystywaną w steganografii jest \textit{Dyskretna transformata falkowa} (ang. \textit{Discrete wavelet transform (DWT)}) oraz jej odpowiednik
    pozwalający na bezstratną transformację odwrotną - \textit{Całkowitoliczbowa transformata falkowa} (ang. \textit{Integer wavelet transform (IWT)}).
    % http://home.agh.edu.pl/~jsw/oso/FT.pdf
    Jedną z możliwości transformaty falkowej jest wyodrębnienie części obrazu w której skład wchodzą osobno wysokie i niskie pasma częstotliwości.
    Obraz zostaje kolejno przekształcany wierszami i kolumnami przez filtr dolnoprzepustowy (wykonujący operację uśredniania) i górnoprzepustowy (obliczający różnicę).
    % TU DODAĆ JAKIŚ SZTOSOWY OBRAZ IWT, http://bigwww.epfl.ch/demo/ip/demos/wavelets/
    Po pierwszej iteracji oryginalny obraz zostaje podzielony na sekcje:
    \begin{itemize}
        \item LL - filter dolnoprzepustowy zaaplikowany do wierszy i kolumn
        \item LH - filter dolnoprzepustowy zaaplikowany do wierszy, górnoprzepustowy dla kolumn
        \item HL - filter górnoprzepustowy zaaplikowany do wierszy, dolnoprzepustowy dla kolumn
        \item HH - filter górnoprzepustowy zaaplikowany do wierszy i kolumn.
    \end{itemize}

    % file:///Users/grzegorzkazana/Desktop/Stegano_Ant/ICME05-Lossless20Data20Hiding20Using20Integer20Wavelet20Transform20and20Threshold20Embedding20Technique.pdf
    W artykule XYZ opisano metodę ukrywania bitów wiadomości w najmniej znaczących bitach współczynników odpowiadających pasmom wysokiej częstotliwości.
    Wyniki eksperymentalne, względem innych metod, wykazały znaczący wzrost pojemności obrazu przy zachowaniu tego samego współczynnika szczytowego
    stosunku sygnału do szumu.

    % PRACE WSPOMINAJĄCE O UKRYWANIU W 'COMPLEX REGIONS'
    % file:///Users/grzegorzkazana/Desktop/Stegano_Ant/ASteganographicMethodForImagesByPixelValueDifferencing.pdf
    % file:///Users/grzegorzkazana/Desktop/Stegano_Ant/Ant%20Colony%20Optimization%20ACO%20based%20Data%20Hiding%20in%20Image%20Complex%20Region.pdf
    % file:///Users/grzegorzkazana/Desktop/Stegano_Ant/Edge-based_image_steganography.pdf
    % file:///Users/grzegorzkazana/Desktop/Stegano_Ant/ICME05-Lossless20Data20Hiding20Using20Integer20Wavelet20Transform20and20Threshold20Embedding20Technique.pdf
}

% steganogram - nośnik + ukrywana wiadomość
% steganoanaliza - proces wykrywania i ekstrakcji tekstu jawnego
% stegokanał - kanał przesyłu ukrytej informacji
