
\chapter{Steganografia}\label{chap:steganography}
{
    % - czym jest steganografia
    % - odniesienie do kryptografii
    % - rys historyczny
    %   - pierwsze użycia w starożytności
    %   - przykłady
    % - steganografia cyfrowa (jakiś diagram idk), opis mediów nośnych
    % - steganografia z wykorzystaniem obrazów
    %   - podział na techniki dziedzinowe i przestrzenne
    %   - opis metod (LSB etc)
    %   - vLSB
    %   - wpływ doboru obszaru na pojemność obrazu

    Steganografia jest dziedziną nauki poświęconą ukrywaniu informacji w jawnych kanałach komunikacji.
    Nazwa nauki wywodzi się z języka greckiego, i może być tłumaczona jako 'ukryte pismo' (\textit{steganós} - ukryty, \textit{graphia} - pismo).

    W celu podkreślenia cech oraz charakteru metod steganograficznych często przytaczane jest również pojęcie kryptografii.
    Obiektem zainteresowań kryptografii jest uniemożliwienie zrozumienia treści wiadomości przez osoby postronne,
    pozbawione do niej dostępu. Współcześnie jest to osiągane poprzez stosowanie klucza dzielonego przez osoby zaufane
    (kryptografia symetryczna) lub par kluczy publicznych i prywatnych (kryptografia asymetryczna). Dzięki ich zastosowaniu,
    osoba postronna pomimo dostępu do szyfrogramu nie jest w stanie wydobyć tekstu jawnego.

    Celem technik steganograficznych jest umożliwienie uczestnikom komunikacji przesyłania informacji bez ujawniania faktu istnienia samego przekazu.
    
    % https://www.petitcolas.net/fabien/publications/ieee99-infohiding.pdf
    Pierwszych przykładów zastosowania steganografii można doszukiwać się już w czasach starożytnych.
    Herodotus, w swoim dziele \textit{Dzieje} opisuje historię greckiego polityka Histiajosa,
    który w celu przekazania poufnej informacji wytatuował ją na skalpie zaufanego niewolnika.
    Gdy jego włosy odrosły, został on wysłany w celu doręczenia listu oraz ukrytej wiadomości.
    Do innych przykładów steganografii można zaliczyć również ukrywanie wiadomości w zapisach nutowych,
    stosowanie atramentów sympatycznych lub technikę mikrokropek,
    która przeżyła swój renesans w czasach zimnej i drugiej wojny światowej.
    
    % https://www.researchgate.net/publication/296695903_Watermarking_Security_-_Fundamentals_Secure_Designs_and_Attacks
    % https://dl.acm.org/doi/10.1145/3206004.3206019
    Warto również podkreślić, że kolejnym z zastosowań steganografii są znaki wodne oraz symbole
    pozwalające na identyfikację źródła informacji. Za równo w przypadku multimediów objętych prawami autorskimi
    jak i poufnych dokumentów, ich producent lub organizacja strzegąca ich tajności może umieścić ukryte
    informacje pozwalające wskazać źródło wycieku.
    Podobną technikę stosują producenci drukarek - seria oraz model drukarki może być odzwierciedlona w drukowanym
    dokumencie poprzez układ niewidocznych gołym okiem żółtych kropek. Takie działania mają na celu ułatwienia
    walki z przestępczością polegającą na fałszowaniu dokumentów i banknotów.

    %  STEGANOGRAFIA CYFROWA
    Wraz z wzrostem wykorzystania komputerów do celach multimedialnych oraz rozpowszechnieniu szerokopasmowego
    internetu coraz popularniejsza i bardziej opłacalna staje się steganografia cyfrowa.
    Jej ideą jest wykorzystanie jako medium nośnego różnych rodzajów plików komputerowych
    lub protokołów komunikacji cyfrowej.
    % https://www.researchgate.net/publication/337304290_Covert_Channel_and_Data_Hiding_in_TCPIP
    Przykładem wykorzystania protokołów do celów steganograficznych może być ukrywanie danych w polach kontrolnych ramek TCP/IP,
    kontrolowanie opóźnień między poszczególnymi pakietami lub nawet umyślne powodowanie utrat wybranych pakietów.

    Znacznie prostszą, lecz bardzo rozwiniętą techniką jest wykorzystanie plików multimedialnych takich jak zdjęcia,
    pliku muzyczne i filmy. Do ich szczególnej atrakcyjności jako medium służące do ukrywania informacji przyczynia
    się między innymi ich wszechobecność, duże rozmiary oraz wysoka nadmiarowość.
    Ostatni aspekt w kontekście steganografii ma szczególne znaczenie, gdyż oznacza że modyfikacja pewnej części informacji
    bitowej zawartej w pliku ma niski wpływ na jego końcową treść. Przykładowo, zmiana wartości jednego z kanałów konkretnego
    piksela będzie miało mało zauważalny wpływ na końcowy obraz nawet przez uważnego obserwatora.
    % https://scholarworks.rit.edu/cgi/viewcontent.cgi?article=1305&context=other
    Podobne prawidłowości można również dostrzec w plikach muzycznych - manipulacja zawartością częstotliwości składowych
    będących poza granicą percepcji, czyli poniżej 20Hz i powyżej 20kHz również będzie trudna w detekcji przez subiektywnego odbiorcę.
    Innym przykładem ukrywania informacji w muzyce jest tzw. \textit{backmasking}, polegający na ukrywaniu wiadomości
    możliwych w odbiorze tylko i wyłącznie po odtworzeniu utworu od tyłu. Jednym z pierwszych zespołów przyczyniających się
    do wzrostu popularności powyższych eksperymentów jest \textit{The Beatles}.
 
    %   TECHNIKI UKRYWANIA INFORMACJI W OBRAZACH
    Mimo tego, że jako medium steganograficzne można wykorzystać każdy plik binarny,
    szczególnie dużo uwagi zostało poświęcone cyfrowym obrazom i zdjęciom.
    Pomimo pozornej prostoty powyższego zadania powstało wiele wyrafinowanych metod i technik, różniących się za równo pod kątem
    założeń jak i rezultatów. 
    Pierwszym parametrem mogącym służyć do podziału zaproponowanych metod jest dziedzina w której 
    obraz zostaje poddany analizie. 

    %   TECHNIKI PRZESTRZENNE
    W technikach przestrzennych, obraz jest traktowany jak zbiór punktów (pikseli) umieszczonych w dwuwymiarowym układzie współrzędnych.
    Zaletą tych metod jest ich intuicyjność oraz przystępność, lecz są to również metody bardziej podatne na ataki polegające na wykryciu
    lub zniszczeniu ukrytej wiadomości. % file:///Users/grzegorzkazana/Desktop/Stegano_Ant/AnInclusiveStudyandAnalysis.pdf

    Najbardziej powszechną techniką przestrzenną jest metoda \textit{Least Significant Bit (LSB)}. Jej zastosowanie sprowadza się
    do zastąpienia najmniej znaczącego bitu obrazu będącego nośnikiem informacji bitem wiadomości ukrywanej. W zależności od spadku
    jakości który uznajemy za akceptowalny, możemy wykorzystać \textit{n} najmniej znaczących bitów każdego z kanałów \textit{RGB}.

    Pewnym uszczegółowieniem \textit{LSB} jest metoda \textit{4LSB}. Zakłada ona wykorzystanie dokładnie 4 bitów z każdego bajtu
    obrazu maskującego, co przekłada się na wykorzystanie 50\% pojemności nośnika. Kosztem tak znacznej pojemności jest znaczący spadek
    jakości obrazu oraz ułatwiona steganoanaliza.

    W celu zmniejszenia wykrywalności manipulacji obrazu przy jednoczesnym zachowaniu względnie dużej pojemności steganogramu,
    zaproponowano technikę \textit{Variable Least Significant Bit (VLSB)}. W przeciwieństwie do \textit{LSB} podczas ukrywania danych
    wykorzystywana jest różna ilość bitów obrazu w zależności od położenia piksela.
    Nośnik zostaje podzielony na zadaną ilość sekcji, a następnie dla każdej z nich jest wyznaczana jest ilość bitów które zostaną
    zastąpione tekstem jawnym. Twórcy metody zaproponowali algorytm \textit{Decreasing Distance Decreasing Bits Algorithm (DDDBA)},
    który na podstawie odległości sektora względem piksela referencyjnego (najczęściej środkowy piksel obrazu) wyznacza proporcjonalną
    ilość ukrywanych bitów. Wynikiem działania algorytmu \textit{VLSB} jest obraz, którego środkowa część jest mniej zniekształcona,
    co obniża subiektywne odczucie spadku jakości i pozwala na ukrycie większej ilości danych. % https://www.researchgate.net/publication/250002424_Implementation_of_Variable_Least_Significant_Bits_Steganography_Using_DDDB_Algorithm

    
    

    %   TECHNIKI CZĘSTOTLIWOŚCIOWE
}

% steganogram - nośnik + ukrywana wiadomość
% steganoanaliza - proces wykrywania i ekstrakcji tekstu jawnego
% stegokanał - kanał przesyłu ukrytej informacji
