
\chapter*{Podsumowanie}\label{chap:summary}\addcontentsline{toc}{chapter}{Podsumowanie}
{
    % w podsumowaniu pracy z kolei powinien się pojawić jeszcze krótszy opis wyników - same wnioski

    W pracy omówiono dwie metody pozwalające na wykorzystanie systemów mrówkowych w celach steganograficznych.
    Szczególną uwagę zwrócono na aspekt degradacji obrazu pod wpływem objętości ukrywanych danych, mający kluczowe
    znaczenie dla poufności przekazu informacji.

    Cel pracy został osiągnięty. Zaproponowane i zaimplementowane metody okazały się skuteczne i pozwalają na ukrycie
    dużej ilości danych przy względnie niskim negatywnym wpływie na jakość obrazu. Przestudiowano wpływ doboru
    parametrów systemów mrówkowych oraz przeprowadzono eksperymenty których celem była ocena efektywności metody.
    Przytoczone wyniki badań zawierają zarówno element oceny subiektywnej, jak i wartości miar pozwalających na
    obiektywne porównania. Uzyskane na referencyjnych obrazach wyniki odniesiono do powiązanych prac, i stwierdzono
    konkurencyjność przytoczonych metod.

    % % a co gdyby pozbyć się mrówek?
    Tematem który mógłby być rozwinięciem poniższej pracy, jest zastosowanie innych bądź większej ilości
    metod wstępnej obróbki obrazu. Eksperymenty związane z metodą opartą na wierzchołkach wykazały, że
    przeskalowanie obrazu wejściowego podczas etapu wyznaczania grafu nie ma kluczowego znaczenia przy ocenie
    jakości steganogramu, a pozwala znacząco skrócić czas obliczeń. Nie jest wykluczone, że istnieją inne, bardziej
    wyrafinowane przekształcenia, które pozwalają polepszyć cechy omawianych metod.

    Inną kwestią wartą dodatkowej uwagi, byłoby porównanie uzyskanych wyników z metodami opartymi o te same założenia,
    lecz nie korzystającymi z heurystyk oraz procesów stochastycznych. Interesującym eksperymentem byłoby podzielenie
    obrazu na wskazaną ilość segmentów, deterministyczne wybranie zadanej ilości segmentów o najwyższej wariancji i
    umieszczenie w nich największej ilości danych. Gdyby w taki sposób osiągnięto zbliżone rezultaty, powyższe
    rozwiązanie miałoby zalety w postaci prostszej implementacji i mniejszej złożoności. Mankamentem takiego rozwiązania
    mogłaby być utrata stochastycznego charakteru procesu, a co za tym idzie spadek bezpieczeństwa steganogramu, jak
    omówiono w rozdziale \ref{sec:heuristics-ants}.

    Kolejnym istotnym tematem badań wartym eksploracji, który nie należał do celów tej pracy, jest analiza wrażliwości
    uzyskanych steganogramów na zniekształcenia i manipulacje oraz podatność na ataki steganograficzne. Uzupełnienie
    przytoczonych eksperymentów o rzeczoną analizę pozwoliłoby na pełniejsze porównanie i odniesienie do pozostałych
    metod.
    %
}
